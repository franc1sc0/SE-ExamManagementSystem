% Product Functions

% Define cross reference
\label{sec:product_functions}

% Define a box around each function
\ifcsmacro{boxed}{}{
   \let\endmyboxed\undefined%
   \newenvironment{boxed}
    {\begin{center} \begin{tabular}{|p{0.9\textwidth}|}
    \hline
    }
    { 
    \\\hline
    \end{tabular} 
    \end{center}
    }
}

The major functions of this application will be used by either Faculty or Students
after logging into the system.

\subsection{Shared Functionality}
   \subsubsection{\large Login} \index{login}
   \begin{boxed} % Boxed section
      \begin{description}
         \item[Description:\label{desc:login}]
      This function allows a Faculty member or Student to login to EMS.
      
         \item[Sequence of Actions:]\hspace{10cm}
      \begin{enumerate}
         \item Faculty or Student provides a username and password
         \item A successful attempt allows the Faculty or Student access to EMS
         \item After three unsuccessful attempts the Faculty or Student's access
            is blocked
      \end{enumerate}

         \item[Rationale:]
            All users of EMS must be logged in to use the functionality.
      \end{description}
   \end{boxed} % End boxed section

\subsection{Faculty Functionality}

   \subsubsection{\large Create Exam} \index{exam!create}
   \begin{boxed} % Boxed section
      \begin{description}
         \item[Description:\label{desc:create_exam}]
      This function allows a Faculty member to create programming and
      communication exams. After creation, the exam is viewable by the Students.
      The exam information includes the exam type, exam administration date,
      start and end time of exam, semester offered, location of exam, and
      registration deadline.
      
         \item[Sequence of Actions:]\hspace{10cm}
      \begin{enumerate}
         \item Faculty provides the data for the exam as defined in
            \autoref{def:exam}
         \item The exam is created
         \item Faculty and Student can see exam
      \end{enumerate}

         \item[Rationale:]
      Because the Faculty administer and grades the exams, he or she should
      be the one capable of creating an exam.
      \end{description}
   \end{boxed} % End boxed section

   \subsubsection{\large View Registered Students in Exam} \index{exam!view registered
   Students}
   \begin{boxed} % Boxed section
      \begin{description}
         \item[Description:\label{desc:view_registered}]
      This function allows a Faculty member to display all registered
         Students for an exam. The information displayed includes username,
         Student ID, Student first and last name, results, comments, and total number
         of registered Students. 
         
            \item[Sequence of Actions:]\hspace{10cm}
         \begin{enumerate}
            \item Faculty is logged in and selects the View Registered Students
               in Exam function
            \item Faculty member is presented with exam details of username,
               Student ID, last name, first name, results (registered or not),
               comments, and the total number of Students registered
         \end{enumerate}

            \item[Rational:]
         Faculty should know the details of an upcoming exam before
         administering it.
      \end{description}
   \end{boxed} % End boxed section

   \subsubsection{\large Edit Exams} \index{exam!edit}
   \begin{boxed} % Boxed section
      \begin{description}
         \item[Description:\label{desc:edit_exams}]
      This function allows a Faculty member to edit exam type, date, start time,
      end time, semester offered, location and registration deadline of an
      existing exam.
         
            \item[Sequence of Actions:]\hspace{10cm}
         \begin{enumerate}
            \item Faculty selects the exam to edit
            \item Faculty member edits any part of an exam as defined in
               \ref{def:exam}
            \item The exam information is updated
         \end{enumerate}

            \item[Rational:]
         Faculty should have the capability to edit details of each exam after
         creation.
      \end{description}
   \end{boxed} % End boxed section

   \subsubsection{\large Enter Exam Results} \index{exam!enter result}
   \begin{boxed} % Boxed section
      \begin{description}
            \item[Description:\label{desc:enter_results}]
      This function allows a Faculty member to enter the results of an exam
      based on Student ID.
         
            \item[Sequence of Actions:]\hspace{10cm}
         \begin{enumerate}
               
            \item Faculty selects the exam to enter results
            \item Faculty member enters exam results based on Student ID
            \item The appropriate exam result as defined in \ref{def:record} is updated in the Student record
         \end{enumerate}

            \item[Rational:]
         Because Faculty are responsible for grading each exam, he or she needs
         to be able to update the results after the exam is administered.
      \end{description}
   \end{boxed} % End boxed section

   \subsubsection{\large View Exam Results} \index{exam!view results}
   \begin{boxed} % Boxed section
      \begin{description}
         \item[Description:\label{desc:view_results}]
      This function allows a Faculty member to display the results of an
         exam. Displayed results include username, Student ID, last name, first
         name, exam results (passed, failed or no show) of each Student along
         with the total passed, failed and no show of Students.
         
            \item[Sequence of Actions:]\hspace{10cm}
         \begin{enumerate}
               
            \item Faculty selects the exam to view
            \item Faculty is presented with all Students registered along with
               the results
         \end{enumerate}

            \item[Rational:]
         Faculty should be able to see an overview of the results of each exam.
      \end{description}
   \end{boxed} % End boxed section

   \subsubsection{\large Publish Exams Result} \index{exam!publish result}
   \begin{boxed} % Boxed section
      \begin{description}
         \item[Description:\label{desc:publish_results}]
      This function allows a Faculty member to publish the results of an
         exam. After being published, Students will be able to see their
         results.
         
            \item[Sequence of Actions:]\hspace{10cm}
         \begin{enumerate}
            \item Faculty selects the exam to be published
            \item Students and Faculty can see the exam results
      \end{enumerate}

            \item[Rational:]
         Faculty should have the capability to make the exam
         results available for Student review.
      \end{description}
   \end{boxed} % End boxed section

   \subsubsection{\large View Student records} \index{Student record!view}
   \begin{boxed} % Boxed section
      \begin{description}
         \item[Description:\label{desc:view_records}]
      This function allows a Faculty member to display all Student
         records information including username, Student ID, first name, last
         name, major, email, exam results and exam history.
         
            \item[Sequence of Actions:]\hspace{10cm}
         \begin{enumerate}
            \item Faculty selects to view all Student records
            \item All Student records details as defined in \ref{def:record} are viewable
      \end{enumerate}

            \item[Rational:]
         Faculty should have the capability of viewing all current Student
         records
      \end{description}
   \end{boxed} % End boxed section

   \subsubsection{\large Create Student record} \index{Student record!create}
   \begin{boxed} % Boxed section
      \begin{description}
         \item[Description:\label{desc:create_record}]
      This function allows a Faculty member to create a new Student
         record of username, Student ID, first name, last name, major, email,
         exam results, phone number, address, city, state and zip code.
         
            \item[Sequence of Actions:]\hspace{10cm}
         \begin{enumerate}
            \item The Faculty selects to create a Student record
            \item The Faculty enters the Student record data as defined in
               \ref{def:record}
            \item The new Student record is created
      \end{enumerate}

            \item[Rational:]
         Faculty should be able to create new Student records as new Students
         enroll.
      \end{description}
   \end{boxed} % End boxed section

   \subsubsection{\large Search Student record} \index{Student record!search}
   \begin{boxed} % Boxed section
      \begin{description}
         \item[Description:\label{desc:search_record}]
      This function allows a Faculty member to search Student records by
      username, Student ID, first name, last name, major, email or exam
      enrollment.
         
            \item[Sequence of Actions:]\hspace{10cm}
         \begin{enumerate}
            \item Faculty selects to search Student records
            \item Faculty searches Student records by username, Student ID,
               first name, last name, major, email or exam enrollment
            \item All Student records that meet the criteria are shown
      \end{enumerate}

            \item[Rational:]
         Faculty should be able to search Student records by various criteria.
      \end{description}
   \end{boxed} % End boxed section

   \subsubsection{\large Edit Student record} \index{Student record!edit}
   \begin{boxed} % Boxed section
      \begin{description}
         \item[Description:\label{desc:edit_record}]
      This function allows a Faculty member to edit Student records. Faculty can
      use username or Student ID to locate a Student record and can edit first
      name, last name, major, email, exam results, phone, address, city, state
      or zip code.
         
            \item[Sequence of Actions:]\hspace{10cm}
         \begin{enumerate}
            \item Faculty selects the Student record to edit by username or
               Student ID
            \item Faculty edit the Student record information as defined in
               \ref{def:record}
            \item. The Student record is updated
      \end{enumerate}

            \item[Rational:]
         Faculty should be able to edit Student records after they have been created.
      \end{description}
   \end{boxed} % End boxed section

   \subsubsection{\large Change Student Exam Result} \index{exam!change result}
   \begin{boxed} % Boxed section
      \begin{description}
         \item[Description:\label{desc:change_result}]
      This function allows a Faculty member to change a Student exam results
      based on username or Student ID.
         
            \item[Sequence of Actions:]\hspace{10cm}
         \begin{enumerate}
               
            \item Faculty selects the Student record based on username or
               Student ID
            \item Faculty can edit one or more of the following elements:
            \begin{itemize}
               \item Programming Exam Results
               \item Communication Exam Results
               \item Core Course Results
            \end{itemize}
         \item The Student record is updated with the new information.
      \end{enumerate}

            \item[Rational:]
         Faculty should be able to change, insert, or delete results from exams
         in Student records.
      \end{description}
   \end{boxed} % End boxed section

   \subsubsection{\large View Complete} \index{Student record!complete} \index{exam!complete}
   \begin{boxed} % Boxed section
      \begin{description}
         \item[Description:\label{desc:view_complete}]
      This function displays all the Students who have passed the
         programming, communication, and core exams. The information displayed
         is username, Student ID, first name, last name, major, email, and
         results of exams.
         
            \item[Sequence of Actions:]\hspace{10cm}
         \begin{enumerate}
            \item Faculty selects to view completed Student records
            \item All Student who have completed all programming, communication
               and core exams are displayed
      \end{enumerate}

            \item[Rational:]
         Faculty should be able to see all Students who have completed the
         required exams.
      \end{description}
   \end{boxed} % End boxed section


\subsection{Student Functionality}

\subsubsection{\large Register Exam} \index{exam!register}
\begin{boxed} % Boxed section
      \begin{description}
         \item[Description:\label{desc:register_exam}]
   This function allows a Student to register for an existing exam by entering a
   username, Student ID, first name, last name, major, email, phone number,
   address, city, and zip code.
         
            \item[Sequence of Actions:]\hspace{10cm}
         \begin{enumerate}
            \item Student selects to register for an exam
            \item The Student enters in the same data as his or her Student
               record as defined in \ref{def:record}
            \item The Student is registered for the exam
         \end{enumerate}

            \item[Rational:]
               Students must register before taking the exam.
      \end{description}
   \end{boxed} % End boxed section

   \subsubsection{\large Withdraw Exam} \index{exam!withdraw}
   \begin{boxed} % Boxed section
      \begin{description}
         \item[Description:\label{desc:withdraw_exam}]
      This function allows a Student to withdraw from an existing exam by
      entering his or her username and the exam to be withdrawn from.
         
            \item[Sequence of Actions:]\hspace{10cm}
         \begin{enumerate}
            \item Student selects the exam to be withdrawn from
            \item Student enters username and exam to withdraw
            \item Student is no longer registered for the exam
         \end{enumerate}

            \item[Rational:]
               Students should be able to withdraw from exams that they are
               registered for.
      \end{description}
   \end{boxed} % End boxed section

   \subsubsection{\large View Exam Result} \index{exam!view result}
   \begin{boxed} % Boxed section
      \begin{description}
         \item[Description:\label{desc:student_view}]
      This function allows a Student to view his or her exam results after
      the exam results are published by a Faculty member.
         
            \item[Sequence of Actions:]\hspace{10cm}
         \begin{enumerate}
            \item Student selects to view the exam results
            \item If Faculty published the exam results, Student views the
               results
            \item Otherwise, Student is informed that the results are not
               published
         \end{enumerate}

            \item[Rational:]
         After the exam is completed, each Student needs to be able to view his
         or her own results of the exam.
      \end{description}
   \end{boxed} % End boxed section
