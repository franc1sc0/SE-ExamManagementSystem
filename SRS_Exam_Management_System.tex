%Copyright 2014 Jean-Philippe Eisenbarth
%This program is free software: you can 
%redistribute it and/or modify it under the terms of the GNU General Public 
%License as published by the Free Software Foundation, either version 3 of the 
%License, or (at your option) any later version.
%This program is distributed in the hope that it will be useful,but WITHOUT ANY 
%WARRANTY; without even the implied warranty of MERCHANTABILITY or FITNESS FOR A 
%PARTICULAR PURPOSE. See the GNU General Public License for more details.
%You should have received a copy of the GNU General Public License along with 
%this program.  If not, see <http://www.gnu.org/licenses/>.

%Based on the code of Yiannis Lazarides
%http://tex.stackexchange.com/questions/42602/software-requirements-specification-with-latex
%http://tex.stackexchange.com/users/963/yiannis-lazarides
%Also based on the template of Karl E. Wiegers
%http://www.se.rit.edu/~emad/teaching/slides/srs_template_sep14.pdf
%http://karlwiegers.com
\documentclass{scrreprt}
\usepackage{listings}
\usepackage{underscore}
\usepackage[bookmarks=true]{hyperref}
\usepackage[utf8]{inputenc}
\usepackage[english]{babel}
\usepackage{import}
\usepackage{imakeidx}
\makeindex[intoc]
\hypersetup{
    bookmarks=false,    % show bookmarks bar?
    pdftitle={SRS of Exam Management System},    % title
    pdfauthor={Olaf Alexander, Shravani Yada, Froylan Morales and Francisco
    Diocares} % author
    pdfsubject={Software Requirement Specifications}, % subject of the document
    pdfkeywords={EMS, SRS, specifications, requirements}, % list of keywords
    colorlinks=true,       % false: boxed links; true: colored links
    linkcolor=blue,       % color of internal links
    citecolor=black,       % color of links to bibliography
    filecolor=black,        % color of file links
    urlcolor=purple,        % color of external links
    linktoc=page            % only page is linked
}%
\def\myversion{1.0 }
\date{}
%\title
\usepackage{hyperref}

% Adjust labeling to depth of 3
\setcounter{secnumdepth}{3}


\begin{document}

\begin{flushleft}
    \rule{16cm}{5pt}\vskip1cm
    \begin{bfseries}
        \Huge{SOFTWARE REQUIREMENTS\\ SPECIFICATION}\\
        \vspace{1.0cm}
        for the\\
        \vspace{1.0cm}
        Exam Management System\\
        \vspace{1.0cm}
        \LARGE{Software Engineering - Spring 2016}\\
        \vspace{1.0cm}
        Prepared by: Olaf Alexander, Shravani Yada, Froylan Morales and Francisco Diocares\\
        \vspace{1.0cm}
        Texas State University\\
        \vspace{1.0cm}
        \today\\
    \end{bfseries}
\end{flushleft}

\tableofcontents
\currentpdfbookmark{Contents}{name}

% Redefine \chapter{} to not create a page break
\makeatletter % Keeps LaTeX from stumbling over @ signs
\renewcommand\chapter{\thispagestyle{plain}%
\global\@topnum\z@
\@afterindentfalse
\secdef\@chapter\@schapter}
\makeatother % Resets @ signs to their normal usage in LaTeX.

\chapter{Introduction}

\section{Purpose}
The purpose of this document is to describe the software requirements for the Exam Management System to manage student exam results.
\\
The intended audience of this SRS is composed of at least two entities: the members of the software engineering team who will be creating the system, and the client who will be receiving the system.

\section{Scope}
Exam Management System will be a web application providing access to registered students and faculty. 
Faculty can register new students, update student information, search student information, change results, create new exams, update existing exam information and publish results. Registered students can register for exams, view results and withdraw an exam.


\section{Definitions, acronyms, and abbreviations}
% Definitions, acronyms, and abbreviations

\begin{description}
   \item[EMS] Exam Management System \index{EMS}
   \item[Exams] \index{exam} Either programming, communication, or core exam taken by Students.
      The results of the exams are pass, fail, or incomplete. Each exam must
      have a type, administration date, start time, end time, semester offered,
      location, and registration deadline.
   \item[Student record] \index{Student record} All information pertaining to a specific Student. This
      includes username, Student ID, Student first and last name, Student major,
      email, programming exam result, communication exam result, each core
      course result, phone number, address, city, state, and zip code of the
      Student.
   \item[Student ID] The Student's identification number used by the University.
   \index{Student ID}
\end{description}


\section{References}
% References

The structure of this software requirement specification is taken from the IEEE
Recommended Practice for Software Requirement Specifications, Std 830-1998.
\\ \\
We should also note that we used the following open source template for our SRS
\begin{itemize}
  \item \nolinkurl{https://github.com/Eisenbarth/SRS-Tex/blob/master/srs.tex}
\end{itemize}

\section{Overview}
\input{srs-resources/1-5-Overview-of-the-document.tex}


\chapter{Overall Description}

\section{Product Perspective}
The Exam Management System is an independent and totally self-contained project that is unencumbered by existing systems. The EMS\index{EMS} is expected\index{expected} to be a complete, correct, and consistent web application.

\section{Product Functions}
% Product Functions

% Define cross reference
\label{sec:product_functions}

% Define a box around each function
\ifcsmacro{boxed}{}{
   \let\endmyboxed\undefined%
   \newenvironment{boxed}
    {\begin{center} \begin{tabular}{|p{0.9\textwidth}|}
    \hline
    }
    { 
    \\\hline
    \end{tabular} 
    \end{center}
    }
}

The major functions of this application will be used by either Faculty or Students
after logging into the system.

\subsection{Faculty Functionality}
The following is functionality that applies to Faculty.

   \subsubsection{\large Create Exams} \index{exam!create}
   \begin{boxed} % Boxed section
      \textbf{Description:}
      {\small This function allows a Faculty member to create programming and
         communication exams. After creation, the exam is viewable by the
         Students. The exam information includes the exam type, exam
         administration date, start and end time of exam, semester offered,
         location of exam, and registration deadline.} \\

      \textbf{Sequence of Actions:}
      \begin{enumerate}
            {\small
         \item Faculty provides the data for the exam to be created.
         \item The exam is created.
         \item Faculty and Student can see exam.}
      \end{enumerate}

      \textbf{Rational:}
      {\small Because the Faculty administer and grades the exams, he or she should
      be the one capable of creating an exam.}
   \end{boxed} % End boxed section

   \subsubsection{\large View Registered Students in Exam} \index{exam!view registered
   Students}
   \begin{boxed} % Boxed section
      \textbf{Description:}
      {\small This function allows a Faculty member to display all registered
         Students for an exam. The information displayed includes username,
         Student ID, Student first and last name, results, comments, and total number
         of registered Students.} \\
         
         \textbf{Sequence of Actions:}
         \begin{enumerate}
               {\small
            \item Faculty is logged in and selects the View Registered Students
               in Exam function.
            \item Faculty member is presented with exam details [username,
               Student ID, last name, first name, results (registered or not),
               comments, and the total number of Students registered]}
         \end{enumerate}

         \textbf{Rational:}
         {\small This allows the Faculty to know the details of an upcoming exam before
         administering it. It is helpful to know how much material to prepare,
         how many Students have registered, etc. beforehand for Faculty.}
   \end{boxed} % End boxed section

   \subsubsection{\large Edit Exams} \index{exam!edit}
   \begin{boxed} % Boxed section
      \textbf{Description:}
      {\small This function allows a Faculty member to edit exam information
         that has already been created.}
         
         \textbf{Sequence of Actions:}
         \begin{enumerate}
               {\small
            \item Faculty selects the exam to edit its details.
            \item Faculty member edits the necessary exam data.
            \item The exam data is saved.}
         \end{enumerate}

         \textbf{Rational:}
         {\small Faculty should have the capability to edit details of each exam after
         creation.}
   \end{boxed} % End boxed section

   \subsubsection{\large Enter Exam Results} \index{exam!enter result}
   \begin{boxed} % Boxed section
      \textbf{Description:}
      {\small This function allows a Faculty member to enter the exam results
         according to Student ID.}
         
         \textbf{Sequence of Actions:}
         \begin{enumerate}
               {\small
            \item Faculty selects the exam to update results.
            \item Faculty member enters exam results.
            \item The exam data is saved. }
         \end{enumerate}

         \textbf{Rational:}
         {\small Because Faculty are responsible for grading each exam, he or she needs
         to be able to update the results after the exam is administered.}
   \end{boxed} % End boxed section

   \subsubsection{\large View Exam Results} \index{exam!view results}
   \begin{boxed} % Boxed section
      \textbf{Description:}
      {\small This function allows a Faculty member to display the results of an
         exam.}
         
         \textbf{Sequence of Actions:}
         \begin{enumerate}
               {\small
            \item Faculty selects the exam to view.
            \item Faculty member is presented the details of the exam selected.}
         \end{enumerate}

         \textbf{Rational:}
         {\small It is useful for Faculty to see an overview of exam results and
         statistical significant data to help prepare for future exams.}
   \end{boxed} % End boxed section

   \subsubsection{\large Publish Exams Result} \index{exam!publish result}
   \begin{boxed} % Boxed section
      \textbf{Description:}
      {\small This function allows a Faculty member to publish the results of an
         exam. After being published, Students will be able to see their
         results.}
         
         \textbf{Sequence of Actions:}
         \begin{enumerate}
               {\small
            \item Faculty selects the exam to be published.
            \item Students and Faculty can see the exam results.}
      \end{enumerate}

         \textbf{Rational:}
         {\small The Students who participated in an exam will need to know their
         results, but only after the Faculty has graded the exam and reviewed
         it. Hence, the Faculty should have the capability to make the exam
         results available for Student review.}
   \end{boxed} % End boxed section

   \subsubsection{\large View Student record} \index{Student record!view}
   \begin{boxed} % Boxed section
      \textbf{Description:}
      {\small This function allows a Faculty member to display all Student
         records.}
         
         \textbf{Sequence of Actions:}
         \begin{enumerate}
               {\small
            \item Faculty selects to view all Student records.
            \item All Student records are viewable.}
      \end{enumerate}

         \textbf{Rational:}
         {\small Faculty should have the capability of viewing all current Student
         records}
   \end{boxed} % End boxed section

   \subsubsection{\large Create Student record} \index{Student record!create}
   \begin{boxed} % Boxed section
      \textbf{Description:}
      {\small This function allows a Faculty member to create a new Student
         record. }
         
         \textbf{Sequence of Actions:}
         \begin{enumerate}
               {\small
            \item The Faculty selects to create a Student record.
            \item The Faculty enters the Student record data.
            \item The new Student record is created.}
      \end{enumerate}

         \textbf{Rational:}
         {\small Faculty need to be able to create new Student records as new Students
         enroll in courses and exams}
   \end{boxed} % End boxed section

   \subsubsection{\large Search Student record} \index{Student record!search}
   \begin{boxed} % Boxed section
      \textbf{Description:}
      {\small This function allows a Faculty member to search Student records.}
         
         \textbf{Sequence of Actions:}
         \begin{enumerate}
               {\small
            \item Faculty selects to search Student records.
            \item Faculty searches Student records by their data.
            \item All Student records that meet the criteria are shown.}
      \end{enumerate}

         \textbf{Rational:}
         {\small Because Faculty should be able to view all Student records, there
         should be an option to search the Records as well.}
   \end{boxed} % End boxed section

   \subsubsection{\large Edit Student record} \index{Student record!edit}
   \begin{boxed} % Boxed section
      \textbf{Description:}
      {\small This function allows a Faculty member to edit Student records.}
         
         \textbf{Sequence of Actions:}
         \begin{enumerate}
               {\small
            \item Faculty selects the Student record to edit.
            \item Faculty edit one or more elements of the Student record.
            \item. The Student record is updated with the new information.}
      \end{enumerate}

         \textbf{Rational:}
         {\small Because Faculty is capable of creating Student records, he or she
         should have the availability to edit Student records after they have
      been created.}
   \end{boxed} % End boxed section

   \subsubsection{\large Change Student Exam Result} \index{exam!change result}
   \begin{boxed} % Boxed section
      \textbf{Description:}
      {\small This function allows a Faculty member to change a Student exam
         results.}
         
         \textbf{Sequence of Actions:}
         \begin{enumerate}
               {\small
            \item Faculty selects the exam to change its results.
            \item Faculty can edit one or more of the following elements:
            \begin{itemize}
               \item Programming Exam Results
               \item Communication Exam Results
               \item Core Course Results
            \end{itemize}
         \item The Student record is updated with the new information.}
      \end{enumerate}

         \textbf{Rational:}
         {\small Faculty should be able to change, insert, or delete results from exams
         in Student records.}
   \end{boxed} % End boxed section

   \subsubsection{\large View Complete} \index{Student record!complete} \index{exam!complete}
   \begin{boxed} % Boxed section
      \textbf{Description:}
      {\small This function displays all the Students who have passed the
         programming, communication, and core exams. }
         
         \textbf{Sequence of Actions:}
         \begin{enumerate}
               {\small
            \item Faculty selects to view completed Student records.
            \item All Student who have completed all necessary exams are
               displayed.}
      \end{enumerate}

         \textbf{Rational:}
         {\small Faculty should be able to see all Students who have completed the
         required exams without having to search through all Student records.}
   \end{boxed} % End boxed section


\subsection{Student Functionality}
The following are functionality that applies to Students.

\subsubsection{\large Register Exam} \index{exam!register}
\begin{boxed} % Boxed section
   \textbf{Description:}
   {\small This function allows a Student to register for an existing exam.}
         
         \textbf{Sequence of Actions:}
         \begin{enumerate}
               {\small
            \item Student selects to register for an exam.
            \item The Student enters in the necessary data to register.
            \item The exam record is updated.}
         \end{enumerate}

         \textbf{Rational:}
         {\small Because a Student is required to take exams with certain criteria and
         restrictions (e.g. only offered on specific dates or a certain set of
         core exams) he or she should be responsible for registering for each
      one.}
   \end{boxed} % End boxed section

   \subsubsection{\large Withdraw Exam} \index{exam!withdraw}
   \begin{boxed} % Boxed section
      \textbf{Description:}
      {\small This function allows a Student to withdraw from an existing exam.}
         
         \textbf{Sequence of Actions:}
         \begin{enumerate}
               {\small
            \item Student selects the exam to be withdrawn from.
            \item Student enters username and exam to withdraw.
            \item The exam record is updated.
            \item Faculty can view updated record.}
         \end{enumerate}

         \textbf{Rational:}
         {\small As it is the Student's responsibility to register for exams, he or she
         should also be able to withdraw for a registered exam.}
   \end{boxed} % End boxed section

   \subsubsection{\large View Exam Result} \index{exam!view result}
   \begin{boxed} % Boxed section
      \textbf{Description:}
      {\small This function allows a Student to view his or her exam results after
      the exam results are published by a Faculty member.}
         
         \textbf{Sequence of Actions:}
         \begin{enumerate}
               {\small
            \item Student selects to view the exam results that they took.
            \item Exam results are published by Faculty.
            \item Student is able to view results}
         \end{enumerate}

         \textbf{Rational:}
         {\small After the exam is completed and graded, each Student needs to
         be able to view his or her own results of the exam.}
   \end{boxed} % End boxed section


\section{User Characteristics}
% User Characteristics

There are two type of users \index{user} who will use the application:
Faculty members and Students. Because each group has a different use of the
application, they each have separate requirements. \index{user!requirements}

The Faculty members have more functionality within the application because of
their ability to create, edit, and search Student records and exams.

The Students will use the application only for exam related activities. The
application should only allow a Student functionality to register,
withdraw and view exams.

For both types of users, no previous training \index{user!training} or specialization should be
required beyond interacting with a website in order to use the application
correctly.


\section{Constraints}
% 2.4 Constraints

% Add cross-reference
\label{constraints}

Full functionality of this application will be available only for registered faculty and students. The GUI presented will only be in english and the system will be working on a single server.For this application to run adequately, it will need to be launched on a system that meets or exceeds the requirements of a 2.0Ghz CPU and 4GB of RAM.




\section{Assumptions and Dependencies}
% 2.5 Assumptions and Dependencies:

EMS will be a web application and it will be available to users with Internet access.



\chapter{Specific Requirements}

\section{External interface requirements}
% 3.1 Interfaces

For this project we will have no interaction external system. Users will have interaction with the system via the menus labeled with possible actions. Data storage and retrieval will be set trough forms, which will have input validation. 
Queries ,inserts, updates and deletes will all be handled by the system.
Students and faculty will have different access to the system and different levels of access to the data, which will be presented to them.


\subsection{User Interfaces}

\subsection{Hardware Interfaces}
% 3.1.2 Hardware Interfaces
\begin{itemize}
  \item A web server will host the EMS application. 
  \item A database server will be used to store student and faculty information.
  \item User will use a device with internet connection to use the EMS application.
\end{itemize}  
  


%\subsection{Software Interfaces}
%% 3.1.3 Software Interfaces
  \begin{itemize}
    \item DBMS \index{DBMS}: TBD
    \item Programming Language \index{programming language}: TBD
    \item Frameworks \index{framework}: TBD
    \item Web Server \index{web server}: TBD
  \end{itemize}


\subsection{Communications Interfaces}
3.1.4 Communication Interfaces
Protocols \index{protocols}
- HTTP \index{HTTP}
- TCP/IP \index{TCP/IP}


% FILE: 3.2_Functional_Requirements.tex
%
% Section 3.2 based off A.3 Template of SRS Section 3 and the Form Based
% specification found on slide 91 of Lecture3.ppt from Spring 2016 CS5391.
%
% All application functionality is organized by user class with each functional
% requirement described by the following criteria:
%  FUNCTION
%  DESCRIPTION
%  INPUTS
%  SOURCES
%  OUTPUTS
%  DESTINATION
%  REQUIRES
%  PRE-CONDITION
%  POST-CONDITION

\section{Functional Requirements}
The following are the functional requirements of the application based on the
user class of Faculty and Student.
\subsection{Faculty User Class}
\subsubsection{Create Exams} \index{exam!create}
\begin{quote} %Indent Section
\begin{description}
\item[Function]
   Create Exams
\item[Description]
   This function allows a Faculty member to create programming and communication
   exams. After creation, the exam is viewable by the Students in their
   interfaces.
\item[Inputs]
   A new exam object.
\item[Sources]
   All inputs are inputted by the Faculty user during exam creation.
\item[Outputs]
   An exam object.
\item[Destination]
   The exam object is committed to the database.
\item[Requires]
   All necessary data is obtained for the new exam object.
\item[Pre-condition]
   The Faculty is logged into the application.
\item[Post-condition]
   The database now has the newly created exam object.
\end{description}
\end{quote} % End indentation

\subsubsection{View Registered Students in Exam} \index{exam!view registered
Students}
\begin{quote} %Indent Section
\begin{description}
\item[Function]
   View Registered Students in Exam
\item[Description]
   This function allows a Faculty member to display all registered Students for
   an exam along with the details of the exam.
\item[Inputs]
   An exam object.
\item[Sources]
   The Faculty selects the exam object.
\item[Outputs]
   The specific exam object's details are returned.
\item[Destination]
   The exam object's details are displayed on the screen
\item[Requires]
   The exam must exist in the database.
\item[Pre-condition]
   The Faculty is logged into the application and has a list of valid exam objects
   to choose from.
\item[Post-condition]
   The specific exam object is retrieved from the database and its details are
   displayed.
\end{description}
\end{quote} % End indentation

\subsubsection{Edit Exams} \index{exam!edit}
\begin{quote} %Indent Section
\begin{description}
\item[Function]
   Edit Exams
\item[Description]
   This function allows a Faculty member to edit exam information that are
   already created.
\item[Inputs]
   An exam object.
\item[Sources]
   The specific exam object is selected by the Faculty.
\item[Outputs]
   The exam object with modified attributes.
\item[Destination]
   The exam object attributes are updated in the database.
\item[Requires]
   The exam object to be in the database and modifiable.
\item[Pre-condition]
   The Faculty is logged into the application and has a list of valid exam objects
   to choose from.
\item[Post-condition]
   The specific exam object's attributes are updated in the database.
\end{description}
\end{quote} % End indentation

\subsubsection{Enter Exam Results} \index{exam!enter results}
\begin{quote} %Indent Section
\begin{description}
\item[Function]
   Enter Exam Results
\item[Description]
   This function allows a Faculty member to enter the exam results
   according to Student ID.
\item[Inputs]
   An exam object and list of Student record objects.
\item[Sources]
   The exam object is selected by the Faculty and the database gets the Student
   records that have registered for the exam according to the attribute
   Student ID.
\item[Outputs]
   The list of Student record objects' attribute of exam result is updated with
   either pass, fail, or noshow. The Faculty will indicate the exam result for each
   Student record.
\item[Destination]
   Each Student record object is updated in the database.
\item[Requires]
   The database to select the correct Student record objects based on Student ID.
\item[Pre-condition]
   The Faculty is logged into the application and has a list of valid exam objects
   to choose from.
\item[Post-condition]
   The Student record objects are updated in the database.
\end{description}
\end{quote} % End indentation

\subsubsection{Publish Exams Result} \index{exam!publish result}
\begin{quote} %Indent Section
\begin{description}
\item[Function]
   Publish Exams Result
\item[Description]
   This function allows a Faculty member to publish the results of an exam.
   After being published, Students will be able to see their results in their
   interface.
\item[Inputs]
   An exam object list of Student record objects.
\item[Sources]
   The exam object is selected by the Faculty and the Student record objects
   from the database.
\item[Outputs]
   The Student record objects attribute of exam results are now viewable by the
   appropriate Students.
\item[Destination]
   The Student record objects are updated in the database.
\item[Requires]
   The correct Student record objects are selected from the database.
\item[Pre-condition]
   The Faculty is logged into the application and has a list of exams objects to
   choose from.
\item[Post-condition]
   The Student record objects are updated such that the exam result attributes
   are viewable.
\end{description}
\end{quote} % End indentation

\subsubsection{View Student record} \index{Student record!view}
\begin{quote} %Indent Section
\begin{description}
\item[Function]
   View Student record
\item[Description]
   This function allows a Faculty member to display all Student records.
\item[Inputs]
   None.
\item[Sources]
   The Faculty selects this function.
\item[Outputs]
   All Student records in the database are returned.
\item[Destination]
   The Student record details are displayed on the screen.
\item[Requires]
   The Student records to be retrievable from the database.
\item[Pre-condition]
   The Faculty is logged into the application.
\item[Post-condition]
   All Student records' details are displayed on the screen.
\end{description}
\end{quote} % End indentation

\subsubsection{Create Student record} \index{Student record!create}
\begin{quote} %Indent Section
\begin{description}
\item[Function]
   Create Student record
\item[Description]
   This function allows a Faculty member to create a new Student record.
\item[Inputs]
   A new Student record object.
\item[Sources]
   The Student record attributes is inputted by the Faculty.
\item[Outputs]
   The Student record object.
\item[Destination]
   The Student record object is created in the database.
\item[Requires]
   All information necessary for a Student record object.
\item[Pre-condition]
   The Faculty is logged into the application and is able to input all necessary
   attributes for the Student record object.
\item[Post-condition]
   A New Student record object is created in the database.
\end{description}
\end{quote} % End indentation

\subsubsection{Search Student record} \index{Student record!search}
\begin{quote} %Indent Section
\begin{description}
\item[Function]
   Search Student record
\item[Description]
   This function allows a Faculty member to search Student records using
   Student ID, username, Student first or last name, major, email, programming
   exam, communication exam, or core course as criteria.
\item[Inputs]
   A list of one or more Student record attributes and Student records.
\item[Sources]
   The list of attributes are inputted by the Faculty and the Student records
   come from the database.
\item[Outputs]
   The Student records who have the matching attributes are returned from the
   database.
\item[Destination]
   The Student records are displayed on the screen.
\item[Requires]
   Student record attributes and access to the database.
\item[Pre-condition]
   The Faculty is logged into the application and is able to input the search
   criteria.
\item[Post-condition]
   A list of matching Student records is displayed on the screen.
\end{description}
\end{quote} % End indentation

\subsubsection{Edit Student record} \index{Student record!edit}
\begin{quote} %Indent Section
\begin{description}
\item[Function]
   Edit Student record
\item[Description]
   This function allows a Faculty member to edit Student records by entering
   Student ID or username to find the Student record and edit the Student
   record's attributes.
\item[Inputs]
   The attribute of a Student record of Student ID or username and a list of all
   Student records.
\item[Sources]
   The search criteria of Student ID or username is inputted by the user and the
   database for the list of Student records.
\item[Outputs]
   The Faculty updates the matching Student record's attributes and the updated
   Student record is returned.
\item[Destination]
   The Student record is updated in the database.
\item[Requires]
   The matching Student record is modifiable in the database.
\item[Pre-condition]
   The Faculty is logged into the system and is able to select the appropriate
   Student record.
\item[Post-condition]
   The Student record is updated and saved in the database.
\end{description}
\end{quote} % End indentation

\subsubsection{Change Student Exam Result \index{exam!change result}}
\begin{quote} %Indent Section
\begin{description}
\item[Function]
   Change Student Exam Result
\item[Description]
   This function allows a Faculty member to change a Student exam results by
   entering his/her Student ID or username, and the results of programming,
   communication, and each core course can be changed.
\item[Inputs]
   The attribute of a Student record of Student ID or username and a list of all
   Student records.
\item[Sources]
   The search criteria of Student ID or username is inputted by the user and the
   database for the list of Student records.
\item[Outputs]
   The Faculty updates the matching Student record's attribute of exam result
   and the updated Student record is returned.
\item[Destination]
   The Student record is updated in the database.
\item[Requires]
   The matching Student record is modifiable in the database.
\item[Pre-condition]
   The Faculty is logged into the system and is able to select the appropriate
   Student record.
\item[Post-condition]
   The Student record is updated and saved in the database.
\end{description}
\end{quote} % End indentation

\subsubsection{View Complete} \index{Student record!complete} \index{exam!complete}
\begin{quote} %Indent Section
\begin{description}
\item[Function]
   View Complete
\item[Description]
   This function displays all the Students who have passed the programming,
   communication, and core exams. The information displayed includes Student ID,
   username, Student first and last name, major, email, result of programming,
   result of communication, result of core.
\item[Inputs]
   None.
\item[Sources]
   The Faculty selects this function.
\item[Outputs]
   A list of Student record objects that have passed the programming,
   communication, and core exams is returned from the database.
\item[Destination]
   The list of matching Student records is displayed on the screen.
\item[Requires]
   The database is accessible.
\item[Pre-condition]
   The Faculty is logged into the application.
\item[Post-condition]
   The list of Student record objects that meet the criteria is displayed on the
   screen.
\end{description}
\end{quote} % End indentation

\subsection{Student User Class}
\subsubsection{Register Exam} \index{exam!register}
\begin{quote} %Indent Section
\begin{description}
\item[Function]
   Register Exam
\item[Description]
   This function allows a Student to register for an existing exam by entering a
   username, Student ID, first and last name, major, email, local phone, local
   address, local city, local state, local zip.  Once a Student has registered
   for the exam, a Faculty member can see this registration in the View
   Registered function.
\item[Inputs]
   A exam object and a Student record object.
\item[Sources]
   The Student record object is inputted by the Student and the exam object is
   obtained from the database.
\item[Outputs]
   The exam object is updated with the Student record now being registered for
   it.
\item[Destination]
   The exam object is updated in the database.
\item[Requires]
   The Student is able to select a list of exam objects.
\item[Pre-condition]
   The Student is logged into the system.
\item[Post-condition]
   The updated exam object is saved in the database.
\end{description}
\end{quote} % End indentation

\subsubsection{Withdraw Exam} \index{exam!withdraw}
\begin{quote} %Indent Section
\begin{description}
\item[Function]
   Withdraw Exam
\item[Description]
   This function allows a Student to withdraw from an existing exam. The Student
   should give username and the exam to withdraw. Once a Student has withdrawn
   an exam, the record seen by the Faculty should be updated.
\item[Inputs]
   A username and exam object.
\item[Sources]
   The username is inputted by the Student and the exam object is obtained from
   the database.
\item[Outputs]
   The exam object is updated with the Student record showing withdrawn.
\item[Destination]
   The exam object is updated in the database.
\item[Requires]
   The Student is able to access a list of exam objects from the database.
\item[Pre-condition]
   The Student is logged into the application and is already registered for an
   exam.
\item[Post-condition]
   The exam object is updated in the database.
\end{description}
\end{quote} % End indentation

\subsubsection{View Exam Result} \index{exam!view result}
\begin{quote} %Indent Section
\begin{description}
\item[Function]
   View Exam Result
\item[Description]
   This function allows a Student to view his or her exam results after the exam
   results are published by a Faculty member.
\item[Inputs]
   None.
\item[Sources]
   The Student selects this function.
\item[Outputs]
   The results of the exam is returned.
\item[Destination]
   The results are displayed on the screen.
\item[Requires]
   The exam results have been published by the Faculty.
\item[Pre-condition]
   The Student is logged into the system.
\item[Post-condition]
   The results of the exam are displayed on the screen.
\end{description}
\end{quote} % End indentation


\section{Performance Requirements}

\section{Design Constraints}
% 3.4 Design constraints 

We will design this program to be as flexible as possible and platform independant.We will implement read
write contraints for the data on the type of user or role. Resources will have to be provided by the client if 
they wish to scale the program further than the tested capacity.There seems not to be other design constraints
imposed on the implementation of this program.


\section{Software System Attributes}
% Software system attributes

The requirements in this section specify the required attributes of performance,
security, and maintainability of the Exam Management System.

\subsubsection{Performance}

The computational and database resources needed for the application are minimal,
and thus a responsive system should be expected. All database calls should be
completed in under a second as long as the server the application is on is not
under a heavy load. Any user visualizations should be updated or presented
within a second as well given the server's overall availability.

\subsubsection{Security}

The application is designed under the assumption that each faculty or student
maintains their username and password in a secure manner. The login function
will allow only valid users access to the appropriate functionality within the
application. Hence, one student should not be able to use the application as
another student or faculty. Because the application will be run on a server,
its overall security is dependant on the server itself and is outside of the
scope of this document.

\subsubsection{Maintainability}

The design and implementation of the application should be done in such a way
that maintainability is easy to extend and refine. This is achieved by a
modular design using commercial, freely-available tools to implement the
application. Test environments should be built for the application to help with
the maintenance of the code and functionality.


%\section{Other Requirements}
%\input{srs-resources/3-6-Other-Requirements.tex}

\newpage
\printindex

\end{document}
