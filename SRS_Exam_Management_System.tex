%Copyright 2014 Jean-Philippe Eisenbarth
%This program is free software: you can 
%redistribute it and/or modify it under the terms of the GNU General Public 
%License as published by the Free Software Foundation, either version 3 of the 
%License, or (at your option) any later version.
%This program is distributed in the hope that it will be useful,but WITHOUT ANY 
%WARRANTY; without even the implied warranty of MERCHANTABILITY or FITNESS FOR A 
%PARTICULAR PURPOSE. See the GNU General Public License for more details.
%You should have received a copy of the GNU General Public License along with 
%this program.  If not, see <http://www.gnu.org/licenses/>.

%Based on the code of Yiannis Lazarides
%http://tex.stackexchange.com/questions/42602/software-requirements-specification-with-latex
%http://tex.stackexchange.com/users/963/yiannis-lazarides
%Also based on the template of Karl E. Wiegers
%http://www.se.rit.edu/~emad/teaching/slides/srs_template_sep14.pdf
%http://karlwiegers.com
\documentclass{scrreprt}
\usepackage{listings}
\usepackage{underscore}
\usepackage[bookmarks=true]{hyperref}
\usepackage[utf8]{inputenc}
\usepackage[english]{babel}
\usepackage{import}
\usepackage{imakeidx}
\makeindex[intoc]
\hypersetup{
    bookmarks=false,    % show bookmarks bar?
    pdftitle={Software Requirement Specification},    % title
    pdfauthor={Jean-Philippe Eisenbarth},                     % author
    pdfsubject={TeX and LaTeX},                        % subject of the document
    pdfkeywords={TeX, LaTeX, graphics, images}, % list of keywords
    colorlinks=true,       % false: boxed links; true: colored links
    linkcolor=blue,       % color of internal links
    citecolor=black,       % color of links to bibliography
    filecolor=black,        % color of file links
    urlcolor=purple,        % color of external links
    linktoc=page            % only page is linked
}%
\def\myversion{1.0 }
\date{}
%\title
\usepackage{hyperref}



\begin{document}

\begin{flushleft}
    \rule{16cm}{5pt}\vskip1cm
    \begin{bfseries}
        \Huge{SOFTWARE REQUIREMENTS\\ SPECIFICATION}\\
        \vspace{1.0cm}
        for the\\
        \vspace{1.0cm}
        Exam Management System\\
        \vspace{1.0cm}
        \LARGE{Software Engineering - Spring 2016}\\
        \vspace{1.0cm}
        Prepared by: Olaf Alexander, Shravani Yada, Froylan Morales and Francisco Diocares\\
        \vspace{1.0cm}
        Texas State University\\
        \vspace{1.0cm}
        \today\\
    \end{bfseries}
\end{flushleft}

\tableofcontents

% Redefine \chapter{} to not create a page break
\makeatletter % Keeps LaTeX from stumbling over @ signs
\renewcommand\chapter{\thispagestyle{plain}%
\global\@topnum\z@
\@afterindentfalse
\secdef\@chapter\@schapter}
\makeatother % Resets @ signs to their normal usage in LaTeX.

\chapter{Introduction}

\section{Purpose}
The purpose\index{purpose} of this document is to describe the software requirements\index{requirements} for the Exam Management System to manage Student exam results.
\\
The intended audience\index{audience} of this SRS is composed of at least two entities\index{entities}: the members of the software engineering team who will be creating the system, and the client, or faculty of the University, who will be receiving the system.


\section{Scope}
Exam Management System will be a web application providing access to registered Students and Faculty. 
Faculty can register new Students, update Student information, search Student information, change results, create new exams, update existing exam information and publish results. Registered Students can register for exams, view results and withdraw an exam.


\section{Definitions, acronyms, and abbreviations}
% Definitions, acronyms, and abbreviations

   EMS -- Exam Management System \index{EMS}
   Exams \index{exam} -- Either programming, communication, or core exam taken by students.
      The results of the exams are pass, fail, or incomplete. Each exam must
      have a type, administration date, start time, end time, semester offered,
      location, and registration deadline.
   Student Record \index{student record} -- All information pertaining to a specific student. This
      includes username, txstateid, student first and last name, student major,
      email, programming exam result, communication exam result, each core
      course result, phone number, address, city, state, and zip code of the
      student.
   txstateid -- The student's identification number used by the University.
   \index{txstateid}


\section{References}
References

The structure of this software requirement specification is taken from the IEEE
Recommended Practice for Software Requirement Specifications, Std 830-1998.


\section{Overview}
% 1.5 Overview of the document 

In the next two chapters we will include an overall description of the project as well as the specific requirements. 

The overall description will include the projects perspective, user characteristics, product functions, constraints, assumptions and dependencies. 
In the specific requirements section we will include more in detail perspective of functions, performance requirements, design constraints and other software system attributes.



\chapter{Overall Description}

\section{Product Perspective}
The Exam Management System is an independent and totally self-contained project that is unencumbered by existing systems. The EMS\index{EMS} is expected\index{expected} to be a complete, correct, and consistent web application.

\section{Product Functions}
% Product Functions

% Define a box around each function
\newenvironment{boxed}
 {\begin{center} \begin{tabular}{|p{0.9\textwidth}|}
 \hline
 }
 { 
 \\\hline
 \end{tabular} 
 \end{center}
 }

The major functions of this application will be used by either Faculty or Students
after logging into the system. They are the following:

\subsection{Faculty Functionality}

   \subsubsection{\large Create Exams} \index{exam!create}
   \begin{boxed} % Boxed section
      \textbf{Description:}
      {\small This function allows a Faculty member to create programming and
         communication exams. After creation, the exam is viewable by the
         Students. The exam information includes the exam type, exam
         administration date, start and end time of exam, semester offered,
         location of exam, and registration deadline.} \\

      \textbf{Sequence of Actions:}
      \begin{enumerate}
            {\small
         \item Faculty provides the data for the exam to be created.
         \item The exam is created.
         \item Faculty and Student can see exam.}
      \end{enumerate}

      \textbf{Rational:}
      {\small Because the Faculty administer and grades the exams, he or she should
      be the one capable of creating an exam.}
   \end{boxed} % End boxed section

   \subsubsection{\large View Registered Students in Exam} \index{exam!view registered
   Students}
   \begin{boxed} % Boxed section
      \textbf{Description:}
      {\small This function allows a Faculty member to display all registered
         Students for an exam. The information displayed includes username,
         Student ID, Student first and last name, results, comments, and total number
         of registered Students.} \\
         
         \textbf{Sequence of Actions:}
         \begin{enumerate}
               {\small
            \item Faculty is logged in and selects the View Registered Students
               in Exam function.
            \item Faculty member is presented with exam details [username,
               Student ID, last name, first name, results (registered or not),
               comments, and the total number of Students registered]}
         \end{enumerate}

         \textbf{Rational:}
         {\small This allows the Faculty to know the details of an upcoming exam before
         administering it. It is helpful to know how much material to prepare,
         how many Students have registered, etc. beforehand for Faculty.}
   \end{boxed} % End boxed section

   \subsubsection{\large Edit Exams} \index{exam!edit}
   \begin{boxed} % Boxed section
      \textbf{Description:}
      {\small This function allows a Faculty member to edit exam information
         that has already been created.}
         
         \textbf{Sequence of Actions:}
         \begin{enumerate}
               {\small
            \item Faculty selects the exam to edit its details.
            \item Faculty member edits the necessary exam data.
            \item The exam data is saved.}
         \end{enumerate}

         \textbf{Rational:}
         {\small Faculty should have the capability to edit details of each exam after
         creation.}
   \end{boxed} % End boxed section

   \subsubsection{\large Enter Exam Results} \index{exam!enter result}
   \begin{boxed} % Boxed section
      \textbf{Description:}
      {\small This function allows a Faculty member to enter the exam results
         according to Student ID.}
         
         \textbf{Sequence of Actions:}
         \begin{enumerate}
               {\small
            \item Faculty selects the exam to update results.
            \item Faculty member enters exam results.
            \item The exam data is saved. }
         \end{enumerate}

         \textbf{Rational:}
         {\small Because Faculty are responsible for grading each exam, he or she needs
         to be able to update the results after the exam is administered.}
   \end{boxed} % End boxed section

   \subsubsection{\large View Exam Results} \index{exam!view results}
   \begin{boxed} % Boxed section
      \textbf{Description:}
      {\small This function allows a Faculty member to display the results of an
         exam.}
         
         \textbf{Sequence of Actions:}
         \begin{enumerate}
               {\small
            \item Faculty selects the exam to view.
            \item Faculty member is presented the details of the exam selected.}
         \end{enumerate}

         \textbf{Rational:}
         {\small It is useful for Faculty to see an overview of exam results and
         statistical significant data to help prepare for future exams.}
   \end{boxed} % End boxed section

   \subsubsection{\large Publish Exams Result} \index{exam!publish result}
   \begin{boxed} % Boxed section
      \textbf{Description:}
      {\small This function allows a Faculty member to publish the results of an
         exam. After being published, Students will be able to see their
         results.}
         
         \textbf{Sequence of Actions:}
         \begin{enumerate}
               {\small
            \item Faculty selects the exam to be published.
            \item Students and Faculty can see the exam results.}
      \end{enumerate}

         \textbf{Rational:}
         {\small The Students who participated in an exam will need to know their
         results, but only after the Faculty has graded the exam and reviewed
         it. Hence, the Faculty should have the capability to make the exam
         results available for Student review.}
   \end{boxed} % End boxed section

   \subsubsection{\large View Student record} \index{Student record!view}
   \begin{boxed} % Boxed section
      \textbf{Description:}
      {\small This function allows a Faculty member to display all Student
         records.}
         
         \textbf{Sequence of Actions:}
         \begin{enumerate}
               {\small
            \item Faculty selects to view all Student records.
            \item All Student records are viewable.}
      \end{enumerate}

         \textbf{Rational:}
         {\small Faculty should have the capability of viewing all current Student
         records}
   \end{boxed} % End boxed section

   \subsubsection{\large Create Student record} \index{Student record!create}
   \begin{boxed} % Boxed section
      \textbf{Description:}
      {\small This function allows a Faculty member to create a new Student
         record. }
         
         \textbf{Sequence of Actions:}
         \begin{enumerate}
               {\small
            \item The Faculty selects to create a Student record.
            \item The Faculty enters the Student record data.
            \item The new Student record is created.}
      \end{enumerate}

         \textbf{Rational:}
         {\small Faculty need to be able to create new Student records as new Students
         enroll in courses and exams}
   \end{boxed} % End boxed section

   \subsubsection{\large Search Student record} \index{Student record!search}
   \begin{boxed} % Boxed section
      \textbf{Description:}
      {\small This function allows a Faculty member to search Student records.}
         
         \textbf{Sequence of Actions:}
         \begin{enumerate}
               {\small
            \item Faculty selects to search Student records.
            \item Faculty searches Student records by their data.
            \item All Student records that meet the criteria are shown.}
      \end{enumerate}

         \textbf{Rational:}
         {\small Because Faculty should be able to view all Student records, there
         should be an option to search the Records as well.}
   \end{boxed} % End boxed section

   \subsubsection{\large Edit Student record} \index{Student record!edit}
   \begin{boxed} % Boxed section
      \textbf{Description:}
      {\small This function allows a Faculty member to edit Student records.}
         
         \textbf{Sequence of Actions:}
         \begin{enumerate}
               {\small
            \item Faculty selects the Student record to edit.
            \item Faculty edit one or more elements of the Student record.
            \item. The Student record is updated with the new information.}
      \end{enumerate}

         \textbf{Rational:}
         {\small Because Faculty is capable of creating Student records, he or she
         should have the availability to edit Student records after they have
      been created.}
   \end{boxed} % End boxed section

   \subsubsection{\large Change Student Exam Result} \index{exam!change result}
   \begin{boxed} % Boxed section
      \textbf{Description:}
      {\small This function allows a Faculty member to change a Student exam
         results.
         
         \textbf{Sequence of Actions:}
         \begin{enumerate}
               {\small
            \item Faculty selects the exam to change its results.
            \item Faculty can edit one or more of the following elements:
            \begin{itemize}
               \item Programming Exam Results
               \item Communication Exam Results
               \item Core Course Results
            \end{itemize}
         \item The Student record is updated with the new information.}
      \end{enumerate}

         \textbf{Rational:}
         {\small Faculty should be able to change, insert, or delete results from exams
         in Student records.}
   \end{boxed} % End boxed section

   \subsubsection{\large View Complete} \index{Student record!complete} \index{exam!complete}
   \begin{boxed} % Boxed section
      \textbf{Description:}
      {\small This function displays all the Students who have passed the
         programming, communication, and core exams. }
         
         \textbf{Sequence of Actions:}
         \begin{enumerate}
               {\small
            \item Faculty selects to view completed Student records.
            \item All Student who have completed all necessary exams are
               displayed.}
      \end{enumerate}

         \textbf{Rational:}
         {\small Faculty should be able to see all Students who have completed the
         required exams without having to search through all Student records.}
   \end{boxed} % End boxed section


\subsection{Student Functionality}

\subsubsection{\large Register Exam} \index{exam!register}
\begin{boxed} % Boxed section
   \textbf{Description:}
   {\small This function allows a Student to register for an existing exam.
         
         \textbf{Sequence of Actions:}
         \begin{enumerate}
               {\small
            \item Student selects to register for an exam.
            \item The Student enters in the necessary data to register.
            \item The exam record is updated.}
         \end{enumerate}

         \textbf{Rational:}
         {\small Because a Student is required to take exams with certain criteria and
         restrictions (e.g. only offered on specific dates or a certain set of
         core exams) he or she should be responsible for registering for each
      one.}
   \end{boxed} % End boxed section

   \subsubsection{\large Withdraw Exam} \index{exam!withdraw}
   \begin{boxed} % Boxed section
      \textbf{Description:}
      {\small This function allows a Student to withdraw from an existing exam.}
         
         \textbf{Sequence of Actions:}
         \begin{enumerate}
               {\small
            \item Student selects the exam to be withdrawn from.
            \item Student enters username and exam to withdraw.
            \item The exam record is updated.
            \item Faculty can view updated record.}
         \end{enumerate}

         \textbf{Rational:}
         {\small As it is the Student's responsibility to register for exams, he or she
         should also be able to withdraw for a registered exam.}
   \end{boxed} % End boxed section

   \subsubsection{\large View Exam Result} \index{exam!view result}
   \begin{boxed} % Boxed section
      \textbf{Description:}
      {\small This function allows a Student to view his or her exam results after
      the exam results are published by a Faculty member.}
         
         \textbf{Sequence of Actions:}
         \begin{enumerate}
               {\small
            \item Student selects to view the exam results that they took.
            \item Exam results are published by Faculty.
            \item Student is able to view results}
         \end{enumerate}

         \textbf{Rational:}
         {\small After the exam is completed and graded, each Student needs to
         be able to view his or her own results of the exam.}
   \end{boxed} % End boxed section


\section{User Characteristics}
% User Characteristics

There are two type of users who will be using the application: Faculty members
and students of the University. Because each group has a different use of the
application, they each have seperate requirements.

The faculty members have more funtionality within the application because of
their ability to create, edit, and search Student Records and Exams. This means
that the application must be efficient in allowing them to search, update and
view information about students and exams.

The students will use the application only for exam related activities. In this
sense, the application should only allow a student limited functionality to
register, withdraw and view exams.

For both types of users, no previous training or specialization should be
required beyond interacting with a website in order to use the application
correctly.


\section{Constraints}
% 2.4 Constraints

% Add cross-reference
\label{sec:constraints}

Full functionality of this application will be available only for registered faculty and students. The GUI presented will only be in english and the system will be working on a single server.For this application to run adequately, it will need to be launched on a system that meets or exceeds the requirements of a 2.0Ghz CPU and 4GB of RAM.




\section{Assumptions and Dependencies}
% 2.5 Assumptions and Dependencies:

Exam Management System \index{EMS} will be a web application and it will be available only to users with Internet access. 
Once the application is deployed on a web server, the user can connect to the application by visiting the website. 
To visit the website, the user is expected to use a compatible web browser (IE 7+, Google Chrome, Mozilla Firefox or similar).



\chapter{Specific Requirements}

\section{External interface requirements}
% 3.1 Interfaces

For this project we will have no interaction external system. Users will have interaction with the system via the menus labeled with possible actions. Data storage and retrieval will be set trough forms, which will have input validation. 
Queries ,inserts, updates and deletes will all be handled by the system.
Students and faculty will have different access to the system and different levels of access to the data, which will be presented to them.


\subsection{User Interfaces}

\subsection{Hardware Interfaces}
% 3.1.2 Hardware Interfaces
- Web server \index{web server}	
  - A system that has a web servlet container to host the application
  - The system will be in a controlled secure environment 
  - Allows students and faculty to use the application by communicating through a specific port number
- Database server \index{database server}
  - A proven reliable database management server (such as Oracle \index{Oracle}, MySQL \index{MySQL}) will be deployed 
  - The system will be in a controlled secure environment 
  - The application will be communicating with the database through a specific port number


\subsection{Software Interfaces}
% 3.1.3 Software Interfaces
- DBMS \index{DBMS}: TBD
  \begin{itemize}
      \item Programming Language \index{programming language}: TBD
      \item Frameworks \index{framework}: TBD
      \item Web Server \index{web server}: TBD
  \end{itemize}


\subsection{Communications Interfaces}
%3.1.4 Communication Interfaces
Protocols: \index{protocols}
\begin{itemize}
  \item HTTP \index{HTTP}
  \item TCP/IP \index{TCP/IP}
\end{itemize}


% FILE: 3.2_Functional_Requirements.tex
%
% section 3.2 based off A.3 Template of SRS section 3 and the Form Based
% specification found on slide 91 of Lecture3.ppt from Spring 2016 CS5391.
%
% All application functionality is organized by user class with each functional
% requirement described by the following criteria:
%  FUNCTION
%  DESCRIPTION
%  INPUTS
%  SOURCES
%  OUTPUTS
%  DESTINATION
%  REQUIRES
%  PRE-CONDITION
%  POST-CONDITION


% Define a box around each function
\ifcsmacro{boxed}{}{
   \let\endmyboxed\undefined%
   \newenvironment{boxed}
    {\begin{center} \begin{tabular}{|p{0.9\textwidth}|}
    \hline
    }
    { 
    \hline
    \end{tabular} 
    \end{center}
    }
}

\section{Functional Requirements}
The functional requirements of the application are based on the user classes of
Faculty and Student.
\subsection{Shared Functional Requirements}

\subsubsection{\large Login} \index{login}
\begin{boxed} % Boxed section
\begin{description}
\item[Description:]
   Refer to the description in \autoref{desc:login}
\item[Inputs:]
   A username and password
\item[Sources:]
   Faculty or Student
\item[Outputs:]
   Valid or invalid attempt
\item[Destination:]
   Database
\item[Requires:]
   None
\item[Pre-condition:]
   Faculty or Student is at the Login Function
\item[Post-condition:]
   Faculty or Student is granted or denied access
\end{description}
\end{boxed} % End boxed section


\subsection{Faculty User Class}

\subsubsection{\large Create Exams} \index{exam!create}
\begin{boxed} % Boxed section
\begin{description}
\item[Description:]
   Refer to the description in \autoref{desc:create_exam}
\item[Inputs:]
   An exam as defined in \autoref{def:exam} Information includes:
   \begin{enumerate}
      \item Type (programming, communication or core exam)
      \item Date
      \item Start time
      \item End time
      \item Semester offered (Fall or Spring)
      \item Location
      \item Registration deadline
   \end{enumerate}
\item[Sources:]
   Faculty
\item[Outputs:]
   A new exam
\item[Destination:]
   Database
\item[Requires:]
   Faculty must be logged in
\item[Pre-condition:]
   The Faculty selects the Create Exams Function
\item[Post-condition:]
   The new exam is created
\end{description}
\end{boxed} % End boxed section

\subsubsection{\large View Registered Students in Exam} \index{exam!view registered
Students}
\begin{boxed} % Boxed section
\begin{description}
\item[Description:]
   Refer to the description in \autoref{desc:view_registered}
\item[Inputs:]
   An exam as defined in \autoref{def:exam}
\item[Sources:]
   Faculty
\item[Outputs:]
   Total students registered, comments and the following information of
   Students:
   \begin{enumerate}
      \item username
      \item Student ID
      \item first name
      \item last name
      \item results (registered/not registered)
   \end{enumerate}
\item[Destination:]
   Screen
\item[Requires:]
   Faculty must be logged in
\item[Pre-condition:]
   The Faculty selects the View Registered Students in Exam Function
\item[Post-condition:]
   The exam details are displayed
\end{description}
\end{boxed} % End boxed section

\subsubsection{\large Edit Exams} \index{exam!edit}
\begin{boxed} % Boxed section
\begin{description}
\item[Description:]
   Refer to the description in \autoref{desc:edit_exams}
\item[Inputs:]
   An exam as defined in \autoref{def:exam}
\item[Sources:]
   Faculty
\item[Outputs:]
   An exam
\item[Destination:]
   Database
\item[Requires:]
   Faculty must be logged in
\item[Pre-condition:]
   The Faculty selects the Edit Exams Function
\item[Post-condition:]
   The specific exam details are updated in the database
\end{description}
\end{boxed} % End boxed section

\subsubsection{\large Enter Exam Results} \index{exam!enter results}
\begin{boxed} % Boxed section
\begin{description}
\item[Description:]
   Refer to the description in \autoref{desc:enter_results}
\item[Inputs:]
   An exam and Student records as defined in \autoref{def:record}
\item[Sources:]
   Faculty (exam) and the database (Student records)
\item[Outputs:]
   Student records
\item[Destination:]
   Database
\item[Requires:]
   Faculty must be logged in
\item[Pre-condition:]
   The Faculty selects the Enter Exam Results Function
\item[Post-condition:]
   The Student records are updated in the database
\end{description}
\end{boxed} % End boxed section

\subsubsection{\large View Exams Results} \index{exam!view results}
\begin{boxed} % Boxed section
\begin{description}
\item[Description:]
   Refer to the description in \autoref{desc:view_results}
\item[Inputs:]
   An exam as defined in \autoref{def:exam}
\item[Sources:]
   Faculty
\item[Outputs:]
   Total students passed, failed and did not show along with the following
   information of Students:
   \begin{enumerate}
      \item username
      \item Student ID
      \item first name
      \item last name
      \item results (passed, failed, no show)
   \end{enumerate}
\item[Destination:]
   Screen
\item[Requires:]
   Faculty must be logged in
\item[Pre-condition:]
   The Faculty selects the View Exams Results Function
\item[Post-condition:]
   The exam results are viewable by Faculty
\end{description}
\end{boxed} % End boxed section

\subsubsection{\large Publish Exams Results} \index{exam!publish results}
\begin{boxed} % Boxed section
\begin{description}
\item[Description:]
   Refer to the description in \autoref{desc:publish_results}
\item[Inputs:]
   An exam as defined in \autoref{def:exam}
\item[Sources:]
   Faculty
\item[Outputs:]
   Student records' exam results (pass, fail, no show) as defined in
   \autoref{def:record}
\item[Destination:]
   Database
\item[Requires:]
   Faculty must be logged in
\item[Pre-condition:]
   The Faculty selects the Publish Exams Results Function
\item[Post-condition:]
   The exam results are viewable by students
\end{description}
\end{boxed} % End boxed section

\subsubsection{\large View Student Records} \index{Student record!view}
\begin{boxed} % Boxed section
\begin{description}
\item[Description:]
   Refer to the description in \autoref{desc:view_records}
\item[Inputs:]
   None
\item[Sources:]
   None
\item[Outputs:]
   All Student records defined in \autoref{def:record}
\item[Destination:]
   Screen
\item[Requires:]
   Faculty must be logged in
\item[Pre-condition:]
   The Faculty selects the View Student Record Function
\item[Post-condition:]
   All Student records' details are displayed
\end{description}
\end{boxed} % End boxed section

\subsubsection{\large Create Student record} \index{Student record!create}
\begin{boxed} % Boxed section
\begin{description}
\item[Description:]
   Refer to the description in \autoref{desc:create_record}
\item[Inputs:]
   All information for a Student record as defined in \autoref{def:record}:
   \begin{enumerate}
      \item username
      \item Student ID
      \item first name
      \item last name
      \item major
      \item email
      \item programming result
      \item communication result
      \item each core exam result
      \item phone number
      \item address
      \item city
      \item state
      \item zip code
   \end{enumerate}
\item[Sources:]
   Faculty
\item[Outputs:]
   A new Student record
\item[Destination:]
   Database
\item[Requires:]
   Faculty must be logged in
\item[Pre-condition:]
   The Faculty selects the Create Student record Function
\item[Post-condition:]
   A New Student record is created
\end{description}
\end{boxed} % End boxed section

\subsubsection{\large Search Student record} \index{Student record!search}
\begin{boxed} % Boxed section
\begin{description}
\item[Description:]
   Refer to the description in \autoref{desc:search_record}
\item[Inputs:]
   One or more attributes of a Student record as defined in \autoref{def:record}
\item[Sources:]
   Faculty
\item[Outputs:]
   Student records with matching attributes
\item[Destination:]
   Screen
\item[Requires:]
   Faculty must be logged in
\item[Pre-condition:]
   The Faculty selects the Search Student record Function
\item[Post-condition:]
   A list of matching Student records is displayed
\end{description}
\end{boxed} % End boxed section

\subsubsection{\large Edit Student record} \index{Student record!edit}
\begin{boxed} % Boxed section
\begin{description}
\item[Description:]
   Refer to the description in \autoref{desc:edit_record}
\item[Inputs:]
   username or Student ID of a Student and one or more of any
   attribute of a Student record as defined in \autoref{def:record}
\item[Sources:]
   Faculty
\item[Outputs:]
   The updated Student record
\item[Destination:]
   Database
\item[Requires:]
   Faculty must be logged in
\item[Pre-condition:]
   The Faculty selects the Edit Student record Function
\item[Post-condition:]
   The Student record that was edited is updated
\end{description}
\end{boxed} % End boxed section

\subsubsection{\large Change Student Exam Result} \index{exam!change result}
\begin{boxed} % Boxed section
\begin{description}
\item[Description:]
   Refer to the description in \autoref{desc:change_result}
\item[Inputs:]
   username or Student Id of a Student and one or more exam results as defined
   in \autoref{def:exam}
\item[Sources:]
   Faculty
\item[Outputs:]
   The updated Student record
\item[Destination:]
   Database
\item[Requires:]
   Faculty must be logged in
\item[Pre-condition:]
   The Faculty selects the Change Student Exam Result Function
\item[Post-condition:]
   The Student record's exam results are updated
\end{description}
\end{boxed} % End boxed section

\subsubsection{\large View Complete} \index{Student record!complete} \index{exam!complete}
\begin{boxed} % Boxed section
\begin{description}
\item[Description:]
   Refer to the description in \autoref{desc:view_complete}
\item[Inputs:]
   Student records as defined in \autoref{def:record}
\item[Sources:]
   Faculty
\item[Outputs:]
   A list of Student records that have passed the programming,
   communication, and core exams
\item[Destination:]
   Screen
\item[Requires:]
   Faculty must be logged in
\item[Pre-condition:]
   The Faculty selects the View Complete Function
\item[Post-condition:]
   The list of Student records that have completed all programming,
   communication, and core exams.
\end{description}
\end{boxed}

\subsection{Student User Class}

\subsubsection{\large Register Exam} \index{exam!register}
\begin{boxed} % Boxed section
\begin{description}
\item[Description:]
   Refer to the description in \autoref{desc:register_exam}
\item[Inputs:]
   An existing exam and a Student record as defined in \autoref{def:exam}
\item[Sources:]
   Student
\item[Outputs:]
   Student record with the exam (programming, communication or core) showing
   registered
\item[Destination:]
   Database
\item[Requires:]
   Student must be logged in
\item[Pre-condition:]
   Student selects the Register Exam Function
\item[Post-condition:]
   The exam shows as registered in the Student record
\end{description}
\end{boxed} % End boxed section

\subsubsection{\large Withdraw Exam} \index{exam!withdraw}
\begin{boxed} % Boxed section
\begin{description}
\item[Description:]
   Refer to the description in \autoref{desc:withdraw_exam}
\item[Inputs:]
   username and exam
\item[Sources:]
   Student
\item[Outputs:]
   Student record with the exam (programming, communication or core) showing
   not registered
\item[Destination:]
   Database
\item[Requires:]
   Student must be logged in
\item[Pre-condition:]
   The Student selects the Withdraw Exam Function
\item[Post-condition:]
   The exam shows as not registered in the Student record
\end{description}
\end{boxed} % End boxed section

\subsubsection{\large View Exam Result} \index{exam!view result}
\begin{boxed} % Boxed section
\begin{description}
\item[Description:]
   Refer to the description in \autoref{desc:student_view}
\item[Inputs:]
   Student's record as defined in \autoref{def:exam}
\item[Sources:]
   Student
\item[Outputs:]
   The Student record exam result
\item[Destination:]
   Screen
\item[Requires:]
   Student must be logged in
\item[Pre-condition:]
   The Student selects the View Exam Result Function
\item[Post-condition:]
   The results of the exam in the Student record are displayed
\end{description}
\end{boxed} % End boxed section


\subsection{Faculty Exam}
Create exam function.

Flow of events	
1. Faculty is logged in and fills in form of Exam type
2. Exam type is created with update ability.
3. Faculty and student can see exam.
Entry Condition	Faculty is logged into EMS
Exit Condition	Exam is created successfully
OR 
Show error message
Quality Requirements:	The Faculty Record should be added to the database in under 1 second and a response should be given to show success/error.Student should be able to see created exam in under 1 second.
	
	
Use Case Name	Faculty Exam Option
Participating actors	 Faculty
Flow of events	
1. Faculty selects Exam Option Function
2. Faculty member is presented with exam creation details [username, student ID, lastname, firstname, results (registered or not), comments, and the total number of students registered ]
Entry Condition	Faculty is logged into EMS
Exit Condition	Extended exam options are presented with editable fields : username, student ID, lastname, firstname, results (registered or not), comments, and the total number of students registered.
OR
Faculty is show error message 
Quality Requirements	Exam option details should be presented under 1 second. Exam Options fields should also be saved under 1 second if updated.
	
Use Case Name	Faculty Edit Exam Option
Participating actors	 Faculty
Flow of events	
1. Faculty selects Edit Option Function
2. Faculty member is presented with editable exam type, exam date, exam state time, exam end time, semester, exam location, registration deadline. 
Entry Condition	Faculty is logged into EMS
Exit Condition	Edit exam options are presented and saved  , redirecting to success page.
OR
Faculty is show error message 
Quality Requirements	Exam edit options should be presented under 1 second. Edit options should be saved in under 1 second if updated.
	
	
Use Case Name	Faculty Enter Results
Participating actors	 Faculty
Flow of events	
1. Faculty selects Enter Results Function
2. Faculty member is presented with form to indicate (noshow, passed, failed) according to student ID. 
Entry Condition	Faculty is logged into EMS
Exit Condition	Exam Results are entered and updated , redirecting to success page.
OR
Faculty is show error message 
Quality Requirements	Results should be saved and reflected only in faculties records in under 1 second.
	
Use Case Name	Faculty View Results
Participating actors	 Faculty
Flow of events	
1. Faculty selects View Result function.
2. Faculty member is presented with username, student ID, last name, first name, results (noshow, passed, failed), and statistics including the total results, how many passed, how many failed, and how many noshow. 
Entry Condition	Faculty is logged into EMS
Exit Condition	Exam Results presented to faculty successfully with exit option.
OR
Faculty is show error message 
Quality Requirements	Results should be shown to faculty in under 1 second.
	
Use Case Name	Faculty Publish Results
Participating actors	 Faculty, Student
Flow of events	
1. Faculty selects Publish results function
2. This function allows a faculty member to publish the exam results.
3 Students can see results.
Entry Condition	Faculty is logged into EMS
Exit Condition	Exam Results saved successfully with page redirect.
OR
Faculty is show error message 
Quality Requirements	Results should be shown to faculty and students in under 1 second.
	


\section{Performance Requirements}

\section{Design Constraints}
% 3.4 Design constraints 



\section{Software System Attributes}
% Software system attributes

% Is this our non-functional requirements section?
The elements in this section specify the required attributes
\index{attributes!requirements} of availability, performance, and security for the EMS.  See \autoref{sec:constraints} for constraints.

\subsection{Availability} \index{availability}

The EMS will be available for use 95\% of the time that the host server running the application is operational.

\subsection{Performance} \index{performance}

All database requests will be completed within five seconds.

\subsection{Security} \index{security}

EMS will use encryption to protect passwords, and system functionality will only be available to users once they have logged in to the system.

% can de remove this section?
%\subsection{Maintainability} \index{maintainability}
%
%All functionality will be documented and each component will be implemented in a
%modular fashion.


\section{Other Requirements}
% Other Requirements



\chapter{Appendixes}


\chapter{References}
References

The structure of this software requirement specification is taken from the IEEE
Recommended Practice for Software Requirement Specifications, Std 830-1998.


\newpage
\printindex

\end{document}
