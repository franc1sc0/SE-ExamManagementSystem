% FILE: 3.2_Functional_Requirements.tex
%
% section 3.2 based off A.3 Template of SRS section 3 and the Form Based
% specification found on slide 91 of Lecture3.ppt from Spring 2016 CS5391.
%
% All application functionality is organized by user class with each functional
% requirement described by the following criteria:
%  FUNCTION
%  DESCRIPTION
%  INPUTS
%  SOURCES
%  OUTPUTS
%  DESTINATION
%  REQUIRES
%  PRE-CONDITION
%  POST-CONDITION


% Define a box around each function
\ifcsmacro{boxed}{}{
   \let\endmyboxed\undefined%
   \newenvironment{boxed}
    {\begin{center} \begin{tabular}{|p{0.9\textwidth}|}
    \hline
    }
    { 
    \hline
    \end{tabular} 
    \end{center}
    }
}

\section{Functional Requirements}
The functional requirements of the application are based on the user class of
Faculty and Student.
\subsection{Faculty User Class}
The following are functionality requirements that apply to Faculty.
\subsubsection{\large Create Exams Function} \index{exam!create}
\begin{boxed} % Boxed section
\begin{description}
\item[Description:]
   This function allows a Faculty member to create programming and communication
   exams. After creation, the exam is viewable by the Students in their
   interfaces.
\item[Inputs:]
   A new exam.
\item[Sources:]
   All inputs are inputted by the Faculty user during exam creation.
\item[Outputs:]
   An exam.
\item[Destination:]
   The exam is committed to the database.
\item[Requires:]
   All necessary data is obtained for the new exam.
\item[Pre-condition:]
   The Faculty selects the Create Exams Function.
\item[Post-condition:]
   The new exam is stored in the database.
\end{description}
\end{boxed} % End boxed section

\subsubsection{\large View Registered Students in Exam Function} \index{exam!view registered
Students}
\begin{boxed} % Boxed section
\begin{description}
\item[Description:]
   This function allows a Faculty member to display all registered Students for
   an exam along with the details of the exam.
\item[Inputs:]
   An exam.
\item[Sources:]
   The Faculty selects the exam.
\item[Outputs:]
   The specific exam details are returned.
\item[Destination:]
   The exam details are displayed on the screen.
\item[Requires:]
   The exam must exist in the database.
\item[Pre-condition:]
   The Faculty selects the View Registered Students in Exam Function.
\item[Post-condition:]
   The specific exam is displayed.
\end{description}
\end{boxed} % End boxed section

\subsubsection{\large Edit Exams Function} \index{exam!edit}
\begin{boxed} % Boxed section
\begin{description}
\item[Description:]
   This function allows a Faculty member to edit exam information that are
   already created.
\item[Inputs:]
   An exam.
\item[Sources:]
   The specific exam is selected by the Faculty.
\item[Outputs:]
   The exam with modified attributes.
\item[Destination:]
   The exam attributes are updated in the database.
\item[Requires:]
   The exam to be in the database and modifiable.
\item[Pre-condition:]
   The Faculty selects the Edit Exams Function.
\item[Post-condition:]
   The specific exam details are updated in the database.
\end{description}
\end{boxed} % End boxed section

\subsubsection{\large Enter Exam Results Function} \index{exam!enter results}
\begin{boxed} % Boxed section
\begin{description}
\item[Description:]
   This function allows a Faculty member to enter the exam results
   according to Student ID.
\item[Inputs:]
   An exam and list of Student record.
\item[Sources:]
   The exam is selected by the Faculty and the database gets the Student
   records that have registered for the exam according to the attribute
   Student ID.
\item[Outputs:]
   The list of Student record attributes of exam result is updated with
   either pass, fail, or no show. The Faculty will indicate the exam result for each
   Student record.
\item[Destination:]
   Each Student record is updated in the database.
\item[Requires:]
   The database to select the correct Student records based on Student ID.
\item[Pre-condition:]
   The Faculty selects the Enter Exam Results Function.
\item[Post-condition:]
   The Student records are updated in the database.
\end{description}
\end{boxed} % End boxed section

\subsubsection{\large Publish Exams Result Function} \index{exam!publish result}
\begin{boxed} % Boxed section
\begin{description}
\item[Description:]
   This function allows a Faculty member to publish the results of an exam.
   After being published, Students will be able to see their results in their
   interface.
\item[Inputs:]
   An exam and list of Student records.
\item[Sources:]
   The exam is selected by the Faculty and the Student records 
   from the database.
\item[Outputs:]
   The Student records attribute of exam results are now viewable by the
   appropriate Students.
\item[Destination:]
   The Student records are updated in the database.
\item[Requires:]
   The correct Student records are selected from the database.
\item[Pre-condition:]
   The Faculty selects the Publish Exams Result Function.
\item[Post-condition:]
   The exam results are viewable by students.
\end{description}
\end{boxed} % End boxed section

\subsubsection{\large View Student record Function} \index{Student record!view}
\begin{boxed} % Boxed section
\begin{description}
\item[Description:]
   This function allows a Faculty member to display all Student records.
\item[Inputs:]
   None.
\item[Sources:]
   The Faculty selects this function.
\item[Outputs:]
   All Student records in the database are returned.
\item[Destination:]
   The Student record details are displayed on the screen.
\item[Requires:]
   The Student records to be retrievable from the database.
\item[Pre-condition:]
   The Faculty selects the View Student record Function.
\item[Post-condition:]
   All Student records' details are displayed.
\end{description}
\end{boxed} % End boxed section

\subsubsection{\large Create Student record Function} \index{Student record!create}
\begin{boxed} % Boxed section
\begin{description}
\item[Description:]
   This function allows a Faculty member to create a new Student record.
\item[Inputs:]
   A new Student record.
\item[Sources:]
   The Student record attributes is inputted by the Faculty.
\item[Outputs:]
   The Student record.
\item[Destination:]
   The Student record is created in the database.
\item[Requires:]
   All information necessary for a Student record.
\item[Pre-condition:]
   The Faculty selects the Create Student record Function.
\item[Post-condition:]
   A New Student record is created.
\end{description}
\end{boxed} % End boxed section

\subsubsection{\large Search Student record Function} \index{Student record!search}
\begin{boxed} % Boxed section
\begin{description}
\item[Description:]
   This function allows a Faculty member to search Student records using
   Student ID, username, Student first or last name, major, email, programming
   exam, communication exam, or core course as criteria.
\item[Inputs:]
   A list of one or more Student record attributes and Student records.
\item[Sources:]
   The list of attributes are inputted by the Faculty and the Student records
   come from the database.
\item[Outputs:]
   The Student records who have the matching attributes are returned from the
   database.
\item[Destination:]
   The Student records are displayed on the screen.
\item[Requires:]
   Student record attributes and access to the database.
\item[Pre-condition:]
   The Faculty selects the Search Student record Function.
\item[Post-condition:]
   A list of matching Student records is displayed.
\end{description}
\end{boxed} % End boxed section

\subsubsection{\large Edit Student record Function} \index{Student record!edit}
\begin{boxed} % Boxed section
\begin{description}
\item[Description:]
   This function allows a Faculty member to edit Student records by entering
   Student ID or username to find the Student record and edit the Student
   record's attributes.
\item[Inputs:]
   The attribute of a Student record of Student ID or username and a list of all
   Student records.
\item[Sources:]
   The search criteria of Student ID or username is inputted by the user and the
   database for the list of Student records.
\item[Outputs:]
   The Faculty updates the matching Student record's attributes and the updated
   Student record is returned.
\item[Destination:]
   The Student record is updated in the database.
\item[Requires:]
   The matching Student record is modifiable in the database.
\item[Pre-condition:]
   The Faculty selects the Edit Student record Function.
\item[Post-condition:]
   The Student record is updated.
\end{description}
\end{boxed} % End boxed section

\subsubsection{\large Change Student Exam Result Function} \index{exam!change result}}
\begin{boxed} % Boxed section
\begin{description}
\item[Description:]
   This function allows a Faculty member to change a Student exam results by
   entering his/her Student ID or username, and the results of programming,
   communication, and each core course can be changed.
\item[Inputs:]
   The attribute of a Student record of Student ID or username and a list of all
   Student records.
\item[Sources:]
   The search criteria of Student ID or username is inputted by the user and the
   database for the list of Student records.
\item[Outputs:]
   The Faculty updates the matching Student record's attribute of exam result
   and the updated Student record is returned.
\item[Destination:]
   The Student record is updated in the database.
\item[Requires:]
   The matching Student record is modifiable in the database.
\item[Pre-condition:]
   The Faculty selects the Change Student Exam Result Function.
\item[Post-condition:]
   The Student record is updated.
\end{description}
\end{boxed} % End boxed section

\subsubsection{\large View Complete Function} \index{Student record!complete} \index{exam!complete}
\begin{boxed} % Boxed section
\begin{description}
\item[Description:]
   This function displays all the Students who have passed the programming,
   communication, and core exams. The information displayed includes Student ID,
   username, Student first and last name, major, email, result of programming,
   result of communication, result of core.
\item[Inputs:]
   None.
\item[Sources:]
   The Faculty selects this function.
\item[Outputs:]
   A list of Student records that have passed the programming,
   communication, and core exams is returned from the database.
\item[Destination:]
   The list of matching Student records is displayed on the screen.
\item[Requires:]
   The database is accessible.
\item[Pre-condition:]
   The Faculty selects the View Complete Function.
\item[Post-condition:]
   The list of Student records that meet the criteria is displayed.

\subsection{Student User Class}
The following are functionality requirements that apply to Students.
\subsubsection{\large Register Exam Function} \index{exam!register}
\begin{boxed} % Boxed section
\begin{description}
\item[Description:]
   This function allows a Student to register for an existing exam by entering a
   username, Student ID, first and last name, major, email, local phone, local
   address, local city, local state, local zip.  Once a Student has registered
   for the exam, a Faculty member can see this registration in the View
   Registered function.
\item[Inputs:]
   An exam and a Student record.
\item[Sources:]
   The Student record is inputted by the Student and the exam is
   obtained from the database.
\item[Outputs:]
   The exam is updated with the Student record now being registered for
   it.
\item[Destination:]
   The exam is updated in the database.
\item[Requires:]
   The Student is able to select a list of exams.
\item[Pre-condition:]
   The Student selects the Register Exam Function.
\item[Post-condition:]
   The updated exam is saved.
\end{description}
\end{boxed} % End boxed section

\subsubsection{\large Withdraw Exam Function} \index{exam!withdraw}
\begin{boxed} % Boxed section
\begin{description}
\item[Description:]
   This function allows a Student to withdraw from an existing exam. The Student
   should give username and the exam to withdraw. Once a Student has withdrawn
   an exam, the record seen by the Faculty should be updated.
\item[Inputs:]
   A username and exam.
\item[Sources:]
   The username is inputted by the Student and the exam is obtained from
   the database.
\item[Outputs:]
   The exam is updated with the Student record showing withdrawn.
\item[Destination:]
   The exam is updated in the database.
\item[Requires:]
   The Student is able to access a list of exams from the database.
\item[Pre-condition:]
   The Student selects the Withdraw Exam Function.
\item[Post-condition:]
   The exam is updated.
\end{description}
\end{boxed} % End boxed section

\subsubsection{\large View Exam Result Function} \index{exam!view result}
\begin{boxed} % Boxed section
\begin{description}
\item[Description:]
   This function allows a Student to view his or her exam results after the exam
   results are published by a Faculty member.
\item[Inputs:]
   None.
\item[Sources:]
   The Student selects this function.
\item[Outputs:]
   The results of the exam is returned.
\item[Destination:]
   The results are displayed on the screen.
\item[Requires:]
   The exam results have been published by the Faculty.
\item[Pre-condition:]
   The Student selects the View Exam Result Function.
\item[Post-condition:]
   The results of the exam are displayed.
\end{description}
\end{boxed} % End boxed section
