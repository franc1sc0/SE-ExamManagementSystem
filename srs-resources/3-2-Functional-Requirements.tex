% FILE: 3.2_Functional_Requirements.tex
%
% Section 3.2 based off A.3 Template of SRS Section 3 and the Form Based
% specification found on slide 91 of Lecture3.ppt from Spring 2016 CS5391.
%
% All application functionality is organized by user class with each functional
% requirement described by the following criteria:
%  FUNCTION
%  DESCRIPTION
%  INPUTS
%  SOURCES
%  OUTPUTS
%  DESTINATION
%  REQUIRES
%  PRE-CONDITION
%  POST-CONDITION

\section{Functional Requirements}
The following are the functional requirements of the application based on the
user class of Faculty and Student.
\subsection{Faculty User Class}
\subsubsection{Create Exams}
\begin{quote} %Indent Section
\begin{description}
\item[Function]
   Create Exams
\item[Description]
   This function allows a faculty member to create programming and communication
   exams. After creation, the exam is viewable by the students in their
   interfaces.
\item[Inputs]
   A new exam object.
\item[Sources]
   All inputs are inputted by the Faculty user during exam creation.
\item[Outputs]
   An exam object.
\item[Destination]
   The exam object is committed to the database.
\item[Requires]
   All necessary data is obtained for the new exam object.
\item[Pre-condition]
   The Faculty is logged into the application.
\item[Post-condition]
   The database now has the newly created exam object.
\end{description}
\end{quote} % End indentation

\subsubsection{View Registered Students in Exam}
\begin{quote} %Indent Section
\begin{description}
\item[Function]
   View Registered Students in Exam
\item[Description]
   This function allows a faculty member to display all registered students for
   an exam along with the details of the exam.
\item[Inputs]
   An exam object.
\item[Sources]
   The Faculty selects the exam object.
\item[Outputs]
   The specific exam object's details are returned.
\item[Destination]
   The exam object's details are displayed on the screen
\item[Requires]
   The exam must exist in the database.
\item[Pre-condition]
   The Faculty is logged into the application and has a list of valid exam objects
   to choose from.
\item[Post-condition]
   The specific exam object is retrieved from the database and its details are
   displayed.
\end{description}
\end{quote} % End indentation

\subsubsection{Edit Exams}
\begin{quote} %Indent Section
\begin{description}
\item[Function]
   Edit Exams
\item[Description]
   This function allows a faculty member to edit exam information that are
   already created.
\item[Inputs]
   An exam object.
\item[Sources]
   The specific exam object is selected by the Faculty.
\item[Outputs]
   The exam object with modified attributes.
\item[Destination]
   The exam object attributes are updated in the database.
\item[Requires]
   The exam object to be in the database and modifiable.
\item[Pre-condition]
   The Faculty is logged into the application and has a list of valid exam objects
   to choose from.
\item[Post-condition]
   The specific exam object's attributes are updated in the database.
\end{description}
\end{quote} % End indentation

\subsubsection{Enter Exam Results}
\begin{quote} %Indent Section
\begin{description}
\item[Function]
   Enter Exam Results
\item[Description]
   This function allows a faculty member to enter the exam results
   according to txstateid.
\item[Inputs]
   An exam object and list of student record objects.
\item[Sources]
   The exam object is selected by the Faculty and the database gets the student
   records that have registered for the exam according to the attribute
   txstateid.
\item[Outputs]
   The list of student record objects' attribute of exam result is updated with
   either pass, fail, or noshow. The Faculty will indicate the exam result for each
   student record.
\item[Destination]
   Each student record object is updated in the database.
\item[Requires]
   The database to select the correct student record objects based on txstateid.
\item[Pre-condition]
   The Faculty is logged into the application and has a list of valid exam objects
   to choose from.
\item[Post-condition]
   The student record objects are updated in the database.
\end{description}
\end{quote} % End indentation

\subsubsection{Publish Exams Result}
\begin{quote} %Indent Section
\begin{description}
\item[Function]
   Publish Exams Result
\item[Description]
   This function allows a faculty member to publish the results of an exam.
   After being published, students will be able to see their results in their
   interface.
\item[Inputs]
   An exam object list of student record objects.
\item[Sources]
   The exam object is selected by the Faculty and the student record objects
   from the database.
\item[Outputs]
   The student record objects attribute of exam results are now viewable by the
   appropriate students.
\item[Destination]
   The student record objects are updated in the database.
\item[Requires]
   The correct student record objects are selected from the database.
\item[Pre-condition]
   The Faculty is logged into the application and has a list of exams objects to
   choose from.
\item[Post-condition]
   The student record objects are updated such that the exam result attributes
   are viewable.
\end{description}
\end{quote} % End indentation

\subsubsection{View Student Record}
\begin{quote} %Indent Section
\begin{description}
\item[Function]
   View Student Record
\item[Description]
   This function allows a faculty member to display all student records.
\item[Inputs]
   None.
\item[Sources]
   The Faculty selects this function.
\item[Outputs]
   All student records in the database are returned.
\item[Destination]
   The student record details are displayed on the screen.
\item[Requires]
   The student records to be retrievable from the database.
\item[Pre-condition]
   The Faculty is logged into the application.
\item[Post-condition]
   All student records' details are displayed on the screen.
\end{description}
\end{quote} % End indentation

\subsubsection{Create Student Record}
\begin{quote} %Indent Section
\begin{description}
\item[Function]
   Create Student Record
\item[Description]
   This function allows a faculty member to create a new student record.
\item[Inputs]
   A new student record object.
\item[Sources]
   The student record attributes is inputted by the Faculty.
\item[Outputs]
   The student record object.
\item[Destination]
   The student record object is created in the database.
\item[Requires]
   All information necessary for a student record object.
\item[Pre-condition]
   The Faculty is logged into the application and is able to input all necessary
   attributes for the student record object.
\item[Post-condition]
   A New student record object is created in the database.
\end{description}
\end{quote} % End indentation

\subsubsection{Search Student Record}
\begin{quote} %Indent Section
\begin{description}
\item[Function]
   Search Student Record
\item[Description]
   This function allows a faculty member to search student records using
   txstateid, username, student first or last name, major, email, programming
   exam, communication exam, or core course as criteria.
\item[Inputs]
   A list of one or more student record attributes and student records.
\item[Sources]
   The list of attributes are inputted by the Faculty and the student records
   come from the database.
\item[Outputs]
   The student records who have the matching attributes are returned from the
   database.
\item[Destination]
   The student records are displayed on the screen.
\item[Requires]
   Student record attributes and access to the database.
\item[Pre-condition]
   The Faculty is logged into the application and is able to input the search
   criteria.
\item[Post-condition]
   A list of matching student records is displayed on the screen.
\end{description}
\end{quote} % End indentation

\subsubsection{Edit Student Record}
\begin{quote} %Indent Section
\begin{description}
\item[Function]
   Edit Student Record
\item[Description]
   This function allows a faculty member to edit student records by entering
   txstateid or username to find the student record and edit the student
   record's attributes.
\item[Inputs]
   The attribute of a student record of txstateid or username and a list of all
   student records.
\item[Sources]
   The search criteria of txstateid or username is inputted by the user and the
   database for the list of student records.
\item[Outputs]
   The Faculty updates the matching student record's attributes and the updated
   student record is returned.
\item[Destination]
   The student record is updated in the database.
\item[Requires]
   The matching student record is modifiable in the database.
\item[Pre-condition]
   The Faculty is logged into the system and is able to select the appropriate
   student record.
\item[Post-condition]
   The student record is updated and saved in the database.
\end{description}
\end{quote} % End indentation

\subsubsection{Change Student Exam Result}
\begin{quote} %Indent Section
\begin{description}
\item[Function]
   Change Student Exam Result
\item[Description]
   This function allows a faculty member to change a student exam results by
   entering his/her txstateid or username, and the results of programming,
   communication, and each core course can be changed.
\item[Inputs]
   The attribute of a student record of txstateid or username and a list of all
   student records.
\item[Sources]
   The search criteria of txstateid or username is inputted by the user and the
   database for the list of student records.
\item[Outputs]
   The Faculty updates the matching student record's attribute of exam result
   and the updated student record is returned.
\item[Destination]
   The student record is updated in the database.
\item[Requires]
   The matching student record is modifiable in the database.
\item[Pre-condition]
   The Faculty is logged into the system and is able to select the appropriate
   student record.
\item[Post-condition]
   The student record is updated and saved in the database.
\end{description}
\end{quote} % End indentation

\subsubsection{View Complete}
\begin{quote} %Indent Section
\begin{description}
\item[Function]
   View Complete
\item[Description]
   This function displays all the students who have passed the programming,
   communication, and core exams. The information displayed includes txstateid,
   username, student first and last name, major, email, result of programming,
   result of communication, result of core.
\item[Inputs]
   None.
\item[Sources]
   The Faculty selects this function.
\item[Outputs]
   A list of student record objects that have passed the programming,
   communication, and core exams is returned from the database.
\item[Destination]
   The list of matching student records is displayed on the screen.
\item[Requires]
   The database is accessible.
\item[Pre-condition]
   The Faculty is logged into the application.
\item[Post-condition]
   The list of student record objects that meet the criteria is displayed on the
   screen.
\end{description}
\end{quote} % End indentation

\subsection{Student User Class}
\subsubsection{Register Exam}
\begin{quote} %Indent Section
\begin{description}
\item[Function]
   Register Exam
\item[Description]
   This function allows a student to register for an existing exam by entering a
   username, txstateid, first and last name, major, email, local phone, local
   address, local city, local state, local zip.  Once a student has registered
   for the exam, a faculty member can see this registration in the View
   Registered function.
\item[Inputs]
   A exam object and a student record object.
\item[Sources]
   The student record object is inputted by the Student and the exam object is
   obtained from the database.
\item[Outputs]
   The exam object is updated with the student record now being registered for
   it.
\item[Destination]
   The exam object is updated in the database.
\item[Requires]
   The Student is able to select a list of exam objects.
\item[Pre-condition]
   The Student is logged into the system.
\item[Post-condition]
   The updated exam object is saved in the database.
\end{description}
\end{quote} % End indentation

\subsubsection{Withdraw Exam}
\begin{quote} %Indent Section
\begin{description}
\item[Function]
   Withdraw Exam
\item[Description]
   This function allows a student to withdraw from an existing exam. The student
   should give username and the exam to withdraw. Once a student has withdrawn
   an exam, the record seen by the faculty should be updated.
\item[Inputs]
   A username and exam object.
\item[Sources]
   The username is inputted by the Student and the exam object is obtained from
   the database.
\item[Outputs]
   The exam object is updated with the student record showing withdrawn.
\item[Destination]
   The exam object is updated in the database.
\item[Requires]
   The student is able to access a list of exam objects from the database.
\item[Pre-condition]
   The Student is logged into the application and is already registered for an
   exam.
\item[Post-condition]
   The exam object is updated in the database.
\end{description}
\end{quote} % End indentation

\subsubsection{View Exam Result}
\begin{quote} %Indent Section
\begin{description}
\item[Function]
   View Exam Result
\item[Description]
   This function allows a student to view his or her exam results after the exam
   results are published by a faculty member.
\item[Inputs]
   None.
\item[Sources]
   The student selects this function.
\item[Outputs]
   The results of the exam is returned.
\item[Destination]
   The results are displayed on the screen.
\item[Requires]
   The exam results have been published by the Faculty.
\item[Pre-condition]
   The Student is logged into the system.
\item[Post-condition]
   The results of the exam are displayed on the screen.
\end{description}
\end{quote} % End indentation
