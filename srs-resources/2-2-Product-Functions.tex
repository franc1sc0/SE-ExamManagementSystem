% Product Functions

% Define a box around each function
\newenvironment{boxed}
 {\begin{center} \begin{tabular}{|p{0.9\textwidth}|}
 \hline
 }
 { 
 \\\hline
 \end{tabular} 
 \end{center}
 }

The major functions of this application will be used by either Faculty or Students
after logging into the system.

\subsection{Faculty Functionality}
The following are functionality that applies to Faculty.

   \subsubsection{\large Create Exams} \index{exam!create}
   \begin{boxed} % Boxed section
      \textbf{Description:}
      {\small This function allows a Faculty member to create programming and
         communication exams. After creation, the exam is viewable by the
         Students. The exam information includes the exam type, exam
         administration date, start and end time of exam, semester offered,
         location of exam, and registration deadline.} \\

      \textbf{Sequence of Actions:}
      \begin{enumerate}
            {\small
         \item Faculty provides the data for the exam to be created.
         \item The exam is created.
         \item Faculty and Student can see exam.}
      \end{enumerate}

      \textbf{Rational:}
      {\small Because the Faculty administer and grades the exams, he or she should
      be the one capable of creating an exam.}
   \end{boxed} % End boxed section

   \subsubsection{\large View Registered Students in Exam} \index{exam!view registered
   Students}
   \begin{boxed} % Boxed section
      \textbf{Description:}
      {\small This function allows a Faculty member to display all registered
         Students for an exam. The information displayed includes username,
         Student ID, Student first and last name, results, comments, and total number
         of registered Students.} \\
         
         \textbf{Sequence of Actions:}
         \begin{enumerate}
               {\small
            \item Faculty is logged in and selects the View Registered Students
               in Exam function.
            \item Faculty member is presented with exam details [username,
               Student ID, last name, first name, results (registered or not),
               comments, and the total number of Students registered]}
         \end{enumerate}

         \textbf{Rational:}
         {\small This allows the Faculty to know the details of an upcoming exam before
         administering it. It is helpful to know how much material to prepare,
         how many Students have registered, etc. beforehand for Faculty.}
   \end{boxed} % End boxed section

   \subsubsection{\large Edit Exams} \index{exam!edit}
   \begin{boxed} % Boxed section
      \textbf{Description:}
      {\small This function allows a Faculty member to edit exam information
         that has already been created.}
         
         \textbf{Sequence of Actions:}
         \begin{enumerate}
               {\small
            \item Faculty selects the exam to edit its details.
            \item Faculty member edits the necessary exam data.
            \item The exam data is saved.}
         \end{enumerate}

         \textbf{Rational:}
         {\small Faculty should have the capability to edit details of each exam after
         creation.}
   \end{boxed} % End boxed section

   \subsubsection{\large Enter Exam Results} \index{exam!enter result}
   \begin{boxed} % Boxed section
      \textbf{Description:}
      {\small This function allows a Faculty member to enter the exam results
         according to Student ID.}
         
         \textbf{Sequence of Actions:}
         \begin{enumerate}
               {\small
            \item Faculty selects the exam to update results.
            \item Faculty member enters exam results.
            \item The exam data is saved. }
         \end{enumerate}

         \textbf{Rational:}
         {\small Because Faculty are responsible for grading each exam, he or she needs
         to be able to update the results after the exam is administered.}
   \end{boxed} % End boxed section

   \subsubsection{\large View Exam Results} \index{exam!view results}
   \begin{boxed} % Boxed section
      \textbf{Description:}
      {\small This function allows a Faculty member to display the results of an
         exam.}
         
         \textbf{Sequence of Actions:}
         \begin{enumerate}
               {\small
            \item Faculty selects the exam to view.
            \item Faculty member is presented the details of the exam selected.}
         \end{enumerate}

         \textbf{Rational:}
         {\small It is useful for Faculty to see an overview of exam results and
         statistical significant data to help prepare for future exams.}
   \end{boxed} % End boxed section

   \subsubsection{\large Publish Exams Result} \index{exam!publish result}
   \begin{boxed} % Boxed section
      \textbf{Description:}
      {\small This function allows a Faculty member to publish the results of an
         exam. After being published, Students will be able to see their
         results.}
         
         \textbf{Sequence of Actions:}
         \begin{enumerate}
               {\small
            \item Faculty selects the exam to be published.
            \item Students and Faculty can see the exam results.}
      \end{enumerate}

         \textbf{Rational:}
         {\small The Students who participated in an exam will need to know their
         results, but only after the Faculty has graded the exam and reviewed
         it. Hence, the Faculty should have the capability to make the exam
         results available for Student review.}
   \end{boxed} % End boxed section

   \subsubsection{\large View Student record} \index{Student record!view}
   \begin{boxed} % Boxed section
      \textbf{Description:}
      {\small This function allows a Faculty member to display all Student
         records.}
         
         \textbf{Sequence of Actions:}
         \begin{enumerate}
               {\small
            \item Faculty selects to view all Student records.
            \item All Student records are viewable.}
      \end{enumerate}

         \textbf{Rational:}
         {\small Faculty should have the capability of viewing all current Student
         records}
   \end{boxed} % End boxed section

   \subsubsection{\large Create Student record} \index{Student record!create}
   \begin{boxed} % Boxed section
      \textbf{Description:}
      {\small This function allows a Faculty member to create a new Student
         record. }
         
         \textbf{Sequence of Actions:}
         \begin{enumerate}
               {\small
            \item The Faculty selects to create a Student record.
            \item The Faculty enters the Student record data.
            \item The new Student record is created.}
      \end{enumerate}

         \textbf{Rational:}
         {\small Faculty need to be able to create new Student records as new Students
         enroll in courses and exams}
   \end{boxed} % End boxed section

   \subsubsection{\large Search Student record} \index{Student record!search}
   \begin{boxed} % Boxed section
      \textbf{Description:}
      {\small This function allows a Faculty member to search Student records.}
         
         \textbf{Sequence of Actions:}
         \begin{enumerate}
               {\small
            \item Faculty selects to search Student records.
            \item Faculty searches Student records by their data.
            \item All Student records that meet the criteria are shown.}
      \end{enumerate}

         \textbf{Rational:}
         {\small Because Faculty should be able to view all Student records, there
         should be an option to search the Records as well.}
   \end{boxed} % End boxed section

   \subsubsection{\large Edit Student record} \index{Student record!edit}
   \begin{boxed} % Boxed section
      \textbf{Description:}
      {\small This function allows a Faculty member to edit Student records.}
         
         \textbf{Sequence of Actions:}
         \begin{enumerate}
               {\small
            \item Faculty selects the Student record to edit.
            \item Faculty edit one or more elements of the Student record.
            \item. The Student record is updated with the new information.}
      \end{enumerate}

         \textbf{Rational:}
         {\small Because Faculty is capable of creating Student records, he or she
         should have the availability to edit Student records after they have
      been created.}
   \end{boxed} % End boxed section

   \subsubsection{\large Change Student Exam Result} \index{exam!change result}
   \begin{boxed} % Boxed section
      \textbf{Description:}
      {\small This function allows a Faculty member to change a Student exam
         results.}
         
         \textbf{Sequence of Actions:}
         \begin{enumerate}
               {\small
            \item Faculty selects the exam to change its results.
            \item Faculty can edit one or more of the following elements:
            \begin{itemize}
               \item Programming Exam Results
               \item Communication Exam Results
               \item Core Course Results
            \end{itemize}
         \item The Student record is updated with the new information.}
      \end{enumerate}

         \textbf{Rational:}
         {\small Faculty should be able to change, insert, or delete results from exams
         in Student records.}
   \end{boxed} % End boxed section

   \subsubsection{\large View Complete} \index{Student record!complete} \index{exam!complete}
   \begin{boxed} % Boxed section
      \textbf{Description:}
      {\small This function displays all the Students who have passed the
         programming, communication, and core exams. }
         
         \textbf{Sequence of Actions:}
         \begin{enumerate}
               {\small
            \item Faculty selects to view completed Student records.
            \item All Student who have completed all necessary exams are
               displayed.}
      \end{enumerate}

         \textbf{Rational:}
         {\small Faculty should be able to see all Students who have completed the
         required exams without having to search through all Student records.}
   \end{boxed} % End boxed section


\subsection{Student Functionality}
The following are functionality that applies to Students.

\subsubsection{\large Register Exam} \index{exam!register}
\begin{boxed} % Boxed section
   \textbf{Description:}
   {\small This function allows a Student to register for an existing exam.}
         
         \textbf{Sequence of Actions:}
         \begin{enumerate}
               {\small
            \item Student selects to register for an exam.
            \item The Student enters in the necessary data to register.
            \item The exam record is updated.}
         \end{enumerate}

         \textbf{Rational:}
         {\small Because a Student is required to take exams with certain criteria and
         restrictions (e.g. only offered on specific dates or a certain set of
         core exams) he or she should be responsible for registering for each
      one.}
   \end{boxed} % End boxed section

   \subsubsection{\large Withdraw Exam} \index{exam!withdraw}
   \begin{boxed} % Boxed section
      \textbf{Description:}
      {\small This function allows a Student to withdraw from an existing exam.}
         
         \textbf{Sequence of Actions:}
         \begin{enumerate}
               {\small
            \item Student selects the exam to be withdrawn from.
            \item Student enters username and exam to withdraw.
            \item The exam record is updated.
            \item Faculty can view updated record.}
         \end{enumerate}

         \textbf{Rational:}
         {\small As it is the Student's responsibility to register for exams, he or she
         should also be able to withdraw for a registered exam.}
   \end{boxed} % End boxed section

   \subsubsection{\large View Exam Result} \index{exam!view result}
   \begin{boxed} % Boxed section
      \textbf{Description:}
      {\small This function allows a Student to view his or her exam results after
      the exam results are published by a Faculty member.}
         
         \textbf{Sequence of Actions:}
         \begin{enumerate}
               {\small
            \item Student selects to view the exam results that they took.
            \item Exam results are published by Faculty.
            \item Student is able to view results}
         \end{enumerate}

         \textbf{Rational:}
         {\small After the exam is completed and graded, each Student needs to
         be able to view his or her own results of the exam.}
   \end{boxed} % End boxed section
