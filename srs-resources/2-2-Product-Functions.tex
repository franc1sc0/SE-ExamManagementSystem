Product Functions

The major functions of this application will be used by either Faculty or Students
after logging into the system. They are the following:

Faculty Functionality

   Create Exams
         This function allows a faculty member to create
         programming and communication exams. After creation, the exam is
         viewable by the students in their interfaces. The exam information
         includes the exam type, exam administration
         date, start and end time of exam, semester offered, location of
         exam, and registration deadline.

         Sequence of Actions:
         1. Faculty is logged in and fills in form of Exam type
         2. Exam type is created with update ability.
         3. Faculty and student can see exam.

         Rationale:
         Because the faculty administer and grades the exams, he or she should
         be the one capable of creating an exam.


   View Registered Students in Exam
         This function allows a faculty member to display all registered
         students for an exam. The information displayed includes username,
         txstateid, student first and last name, results, comments, and total
         number of registered students.
         
         Sequence of Actions:
         1. Faculty is logged in and selects View Registered Students in Exam
            Option.
         2. Faculty member is presented with exam details [username,
            txstateid, last name, first name, results (registered or not), comments,
            and the total number of students registered]

         Rationale:
         This allows the faculty to know the details of an upcoming exam before
         administering it. It is helpful to know how much material to prepare,
         how many students have registered, etc. beforehand for faculty.

   Edit Exams
         This function allows a faculty member to edit exam information that are already
         created. The exam type, date, start time, end time, semester,
         location, and registration deadline can be modified. 
         
         Sequence of Actions:
         1. Faculty is logged in and selects Edit Exams Option.
         2. Faculty member is presented with editable exam type, exam date, exam
            state time, exam end time, semester, exam location, registration
            deadline. 

         Rationale:
         Faculty should have the capability to edit details of each exam after
         creation.

   Enter Exam Results
         This function allows a faculty member to enter the exam results
         according to txstateid.
         
         Sequence of Actions:
         1. Faculty selects Enter Results Function
         2. Faculty member is presented with form to indicate (noshow, passed,
            failed) according to txstateid. 

         Rationale:
         Because faculty are responsible for grading each exam, he or she needs
         to be able to update the results after the exam is administered.

   View Exam Results
         This function allows a faculty member to display the results of an
         exam. Displayed information includes username, txstateid, student
         first and last name, results, and statistics such as total results,
         number of students who passed, number of students who
         failed, and number of students who were not present.
         
         Sequence of Actions:
         1. Faculty selects View Result function.
         2. Faculty member is presented with username, txstateid, last name,
            first name, results (no show, passed, failed), and statistics including
            the total results, how many passed, how many failed, and how many
            no show. 

         Rationale:
         It is useful for Faculty to see an overview of exam results and
         statistical significant data to help prepare for future exams.

   Publish Exams Result
         This function allows a faculty member to publish the results of an
         exam. After being published, students will be able to see their
         results in their interface.
         
         Sequence of Actions:
         1. Faculty selects Publish results function
         2. This function allows a faculty member to publish the exam results.
         3. Students can see results.

         Rationale:
         The students who participated in an exam will need to know their
         results, but only after the faculty has graded the exam and reviewed
         it. Hence, the faculty should have the capability to make the exam
         results available for student review.

   View Student Record
         This function allows a faculty member to display all student records
         which includes username, txstateid, student first and last name,
         student major, email, result of programming exam, result of
         communication exam, result of core courses, and exam history.
         
         Sequence of Actions:
         1. Faculty is logged in and selects the View Student Records Option.
         2. The Student Records are returned.

         Rationale:
         Faculty should have the capability of viewing all current student
         records

   Create Student Record
         This function allows a faculty member to create a new student
         record that includes username, txstateid, student first and last name,
         student major, email, programming exam result, communication exam
         result, each core course result, phone number, address, city, state,
         and zip code of the student.
         
         Sequence of Actions:
         1. The faculty is logged in and selects the Create Student Record
            Option.
         2. A form is presented to get the following information:
               * Texas State ID
               * Username
               * Last Name
               * First Name
               * Major
               * Email
               * Programming Exam Results
               * Communication Exam Results
               * Core Course Results
               * Phone
               * Address
               * City
               * State
               * Zip
         3. The new Student Record is created

         Rationale:
         Faculty need to be able to create new student records as new students
         enroll in courses and exams

   Search Student Record
         This function allows a faculty member to search student records
         using txstateid, username, student first or last name, major,
         email, programming exam, communication exam, or core course as
         criteria.
         
         Sequence of Actions:
         1. Faculty is logged in and selects the Search Student Record Option.
         2. Faculty searches Student Records based on one or more criteria:
               * Texas State ID
               * Username
               * Last Name
               * First Name
               * Major
               * Email
               * Programming Exam Results
               * Communication Exam Results
               * Core Course Results
         3. All Student Records that meet the criteria are shown.

         Rationale:
         Because Faculty should be able to view all Student Records, there
         should be an option to search the Records as well.

   Edit Student Record
         This function allows a faculty member to edit student records by
         entering txstateid or username to find the student and edit the
         student's last name or first name, major, email, programming result,
         communication result, each core course result, local phone, local
         address, local city, local state, local zip.
         
         Sequence of Actions:
         1. Faculty is logged in and selects the Edit Student Record Option.
         2. Faculty uses either username or txstateid to access a specific
            Student Record.
         3. Faculty can edit one or more of the following elements of the
            Student Record:
               * Texas State ID
               * Username
               * Last Name
               * First Name
               * Major
               * Email
               * Programming Exam Results
               * Communication Exam Results
               * Core Course Results
               * Phone
               * Address
               * City
               * State
               * Zip
         4. The Student Record is updated with the new information.

         Rationale:
         Because faculty is capable of creating Student Records, he or she
         should have the availability to edit Student Records after they have
         been created.

   Change Student Exam Result
         This function allows a faculty member to change a student exam
         results by entering his/her txstateid or username, and the results of
         programming, communication, and each core course can be changed.
         
         Sequence of Actions:
         1. Faculty selects the Change Exam Result Option.
         2. Faculty uses either txstateid or user name to access the Student
            Record to edit the Exam Result.
         3. Faculty can edit one or more of the following elements:
               * Programming Exam Results
               * Communication Exam Results
               * Core Course Results
         4. The Student Record is updated with the new information

         Rationale:
         Faculty should be able to change, insert, or delete results from exams
         in Student Records.

   View Complete
         This function displays all the students who have passed the
         programming, communication, and core exams. The information
         displayed includes txstateid, username, student first and last
         name, major, email, result of programming, result of communication,
         result of core.
         
         Sequence of Actions:
         1. Faculty selects the View Completed Option.
         2. All Student who have completed all necessary exams are displayed.

         Rationale:
         Faculty should be able to see all students who have completed the
         required exams without having to search through all Student Records.


Student Functionality

   Register Exam
         This function allows a student to register for an existing exam by
         entering a username, txstateid, first and last name, major, email,
         local phone, local address, local city, local state, local zip.
         Once a student has registered for the exam, a faculty member can see
         this registration in the View Registered function.
         
         Sequence of Actions:
         1. Student is logged in and goes to Register Exam and selects
            Programming/Communcation/Core Course Exam to register.
         2. The Register Exam function accepts the necessary data for the exam
            registration:
               * Texas State ID
               * Last Name
               * First Name
               * Major
               * Email
               * Phone
               * Address
               * City
               * State
               * Zip
         3. Exam record is updated to the database
         4. Faculty member can see this registration in the View Registered
            function.

         Rationale:
         Because a student is required to take exams with certain criteria and
         restrictions (e.g. only offered on specific dates or a certain set of
         core exams) he or she should be responsible for registering for each one.

   Withdraw Exam
         This function allows a student to withdraw from an existing exam. The
         student should give username and the exam to withdraw. Once a
         student has withdrawn an exam, the record seen by the faculty should
         be updated.
         
         Sequence of Actions:
         1. Student is logged in and selects the Withdraw Exam Option.
         2. Student enters username and exam to withdraw.
         2. Exam record is updated.
         3. Faculty can view updated record.

         Rationale:
         As it is the student's responsibility to register for exams, he or she
         should also be able to withdraw for a registered exam.

   View Exam Result
         This function allows a student to view his or her exam results after
         the exam results are published by a faculty member.
         
         Sequence of Actions:
         1. Student is logged in and selects the View Exam Result option.
         2. Exam results are published by faculty
         3. Student is able to view results

         Rationale:
         After the exam is completed and graded, each student needs to be able
         to view his or her own results of the exam.
